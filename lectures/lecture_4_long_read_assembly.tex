\documentclass[11pt,a4paper]{article}

% ===============================
% Packages
% ===============================
\usepackage[margin=2.2cm]{geometry}
\usepackage{setspace}
\usepackage{titlesec}
\usepackage{enumerate}
\usepackage{tabularx}
\usepackage{booktabs}
\usepackage{amsmath}
\usepackage{hyperref}
\usepackage{fancyhdr}
\usepackage{parskip}
\usepackage{lmodern}
\usepackage{xcolor}
\usepackage{environ}

% ===============================
% Toggle solutions (instructor notes)
% ===============================
\newif\ifsolutions
\solutionsfalse    % <-- Student version
%\solutionstrue      % <-- Instructor version

\definecolor{instrblue}{RGB}{0,70,160}

\NewEnviron{solution}{%
  \ifsolutions
    \par\medskip
    \noindent\textbf{\color{instrblue}Instructor note. }\color{instrblue}\BODY
    \par\medskip\color{black}
  \fi
}

\newcommand{\instrnote}[1]{\ifsolutions{\color{instrblue}\emph{[#1]}}\fi}

% ===============================
% Header / footer
% ===============================
\pagestyle{fancy}
\fancyhf{}
\lhead{BINF301 -- Discussion Plan}
\rhead{90-Min Session}
\rfoot{\thepage}
\ifsolutions
  \lhead{Instructor Version -- Notes Included}
\fi

% Section spacing
\titlespacing*{\section}{0pt}{10pt}{4pt}
\titlespacing*{\subsection}{0pt}{7pt}{3pt}

% ===============================
\begin{document}

\begin{center}
    {\LARGE \textbf{90-Minute Discussion Session Plan}}\\[4pt]
    {\Large \textbf{Lecture 4: Long-Read Assembly}}\\[10pt]
    \textbf{Course:} BINF301 -- Computational Biology\\
    \textbf{Instructor:} Tom Michoel \\
    \textbf{Date:} 26/01/2026\\
    \textbf{Created with Copilot}
\end{center}

\vspace{0.5cm}

\section*{0--10 min -- Warm-Up}

\textbf{Prompt:}
\begin{itemize}
    \item ``Which long-read technology (PacBio CLR, PacBio HiFi, Nanopore) do you find most intriguing, and why?''
    \item ``What is one challenge you associate with long-read sequencing?''
\end{itemize}

\begin{solution}
Goal: Activate prior knowledge and normalize uncertainty around long-read technologies.  
Encourage students to distinguish \emph{read length vs. accuracy}, and recall that early long-read technologies were noisy, while HiFi reads dramatically improved accuracy.
\end{solution}

% ------------------------------------------------------------------

\section*{10--25 min -- Think-Pair-Share}

\textbf{Main prompt:}  
Why are De Bruijn graphs difficult to use effectively for long-read assembly?

\textbf{Follow-ups:}
\begin{itemize}
    \item ``What assumption in DBG construction breaks for long reads?''
    \item ``Why is overlap-based assembly more appropriate?''
    \item ``How do read count and error rate affect $k$-mer completeness?''
\end{itemize}

\begin{solution}
Students should recall that DBG assembly requires high and relatively uniform $k$-mer coverage.  
Long-read datasets have \emph{fewer reads} for the same coverage, and historically had \emph{higher error rates}, producing many low-frequency erroneous $k$-mers.  
Therefore long-read assemblers rely on overlap graphs, not DBGs.
\end{solution}

% ------------------------------------------------------------------

\section*{25--45 min -- Structured Group Discussion}

Students form groups of three with the following rotating roles:

\subsection*{Roles}
\begin{itemize}
    \item \textbf{Summarizer:} Explain how PacBio HiFi differs from older PacBio CLR/Nanopore reads and why this affects assembly.
    \item \textbf{Questioner:} Prepare one question about error correction strategies (hybrid, hierarchical, direct).
    \item \textbf{Connector:} Link overlap detection techniques (minimizers, MHAP) to the computational difficulty of all-vs-all overlaps.
\end{itemize}

\subsection*{Starter Question}
How do modern assemblers (Canu, Flye, HiCanu, HiFiAsm) differ in how they treat noise, repeats, and haplotypes?

\begin{solution}
Expected points:
\begin{itemize}
    \item \textbf{Canu}: hierarchical correction; MHAP for fast approximate overlaps.
    \item \textbf{Flye}: repeat graph model; distinguishes bridged vs.\ unbridged repeats.
    \item \textbf{HiCanu}: optimized for HiFi; homopolymer compression; pileup-based correction.
    \item \textbf{HiFiAsm}: haplotype-aware assembly; multi-pass correction and string-graph phasing.
\end{itemize}
Encourage students to compare \emph{overlap graphs}, \emph{repeat graphs}, and \emph{string graphs}.
\end{solution}

% ------------------------------------------------------------------
\section*{\textit{Break (15 min)}}

\medskip

% ------------------------------------------------------------------

\section*{45--65 min -- Exercises}

See handout.

\subsection*{Exercise 1 -- MinHash Overlap Detection}
Sketch how minimizers or MHAP signatures help detect approximate overlaps between long reads.

\subsection*{Exercise 2 -- Bridged vs.\ Unbridged Repeats}
Given a repeat diagram, identify which repeats are bridged by long reads and which require Flye’s repeat-graph method.

\subsection*{Exercise 3 -- Haplotype Bubbles}
Given a string-graph bubble, determine whether it represents heterozygosity or sequencing noise.

\subsection*{Exercise 4 -- Hi-C Scaffolding Clues}
Interpret a schematic Hi-C heatmap showing either a clean diagonal or a diagonal break.

\begin{solution}
  
Instructor should prompt students to think about:
\begin{itemize}
    \item Evidence that a repeat is bridged (a single read spans both copies).
    \item Which SNP-rich reads inform haplotype assignment.
    \item How SALSA2 or YaHS break contigs at low Hi-C support boundaries.
\end{itemize}
Applied work reinforces understanding of algorithms and data interpretation.
\end{solution}

% ------------------------------------------------------------------

\section*{65--80 min -- Synthesis}

\textbf{Discussion prompts:}
\begin{itemize}
    \item ``What makes long-read assembly fundamentally different from short-read assembly?''
    \item ``Which problem does each major tool (Canu, Flye, HiCanu, HiFiAsm) solve particularly well?''
    \item ``If you designed a new assembler, which component would you innovate--overlap detection, repeat resolution, or haplotype phasing?''
\end{itemize}

\begin{solution}
Encourage precise terminology: ``overlap graph,'' ``repeat graph,'' ``string graph,'' ``homopolymer compression,'' ``haplotype bubble.''  
This synthesis stage consolidates algorithmic, biological, and computational aspects.
\end{solution}

% ------------------------------------------------------------------

\section*{80--90 min -- Reflection \& Wrap-Up}

\textbf{Closing prompts:}
\begin{itemize}
    \item ``What is one misconception you corrected today?''
    \item ``Which assembly step (overlap detection, repeat handling, phasing, scaffolding) is still unclear?''
    \item ``What tool or concept would you like a demonstration of in the next session?''
\end{itemize}

\begin{solution}
Reflection helps identify common pain points (phasing, Hi-C interpretation, repeat resolution).  
Use student feedback to calibrate future pacing and choose topics for deeper walkthroughs.
\end{solution}

% ------------------------------------------------------------------


\end{document}