\documentclass[11pt,a4paper]{article}

% ===============================
% Packages
% ===============================
\usepackage[margin=2.2cm]{geometry}
\usepackage{titlesec}
\usepackage{parskip}
\usepackage{lmodern}
\usepackage{fancyhdr}
\usepackage{xcolor}
\usepackage{environ}
\usepackage{enumerate}
\usepackage{hyperref}

% ===============================
% Toggle for instructor notes
% ===============================
\newif\ifsolutions
\solutionsfalse   % <-- Student version
%\solutionstrue     % <-- Instructor version

\definecolor{instrblue}{RGB}{0,70,160}

\NewEnviron{solution}{%
  \ifsolutions
    \par\medskip
    \noindent\textbf{\color{instrblue}Instructor note. }\color{instrblue}\BODY
    \par\medskip\color{black}
  \fi
}

% ===============================
% Header / footer
% ===============================
\pagestyle{fancy}
\fancyhf{}
\lhead{BINF301 -- Discussion Plan}
\rhead{Lecture 3: Genome Assembly (60 min)}
\rfoot{\thepage}
\ifsolutions
  \lhead{Instructor Version -- Notes Included}
\fi

% Section spacing
\titlespacing*{\section}{0pt}{10pt}{4pt}
\titlespacing*{\subsection}{0pt}{7pt}{3pt}

% ===============================
\begin{document}

\begin{center}
    {\LARGE \textbf{60-Minute Discussion Session Plan}}\\[4pt]
    {\Large \textbf{Lecture 7: Searching Genomes and Genome Indexing}}\\[10pt]
    \textbf{Course:} BINF301 -- Computational Biology\\
    \textbf{Instructor:} Tom Michoel\\
    \textbf{Date:} 4/2/2026\\
    \textbf{Created with Copilot}
\end{center}

\vspace{0.5cm}

% ==========================================================
\section*{0-5 min -- Warm-Up}

%\textbf{Purpose:} Reduce anxiety barriers and encourage early participation.


\textbf{Prompts:}
\begin{itemize}
    \item ``When you use Ctrl+F or grep, what do you think happens computationally?''
    \item ``Why might naïve string matching fail for gigabase-scale genomes?''
    \item ``Which concept from the pre-read seemed most confusing: Boyer--Moore, suffix arrays, or FM-index?''
\end{itemize}

\begin{solution}
Goal: surface intuition and break the assumption that search is trivial.  
Anchor discussion in Slides 3--6, which show costly worst-case $O(mn)$ naive search. 
\end{solution}

% ----------------------------------------------
\section*{5--20 min -- Guided Concept Walkthrough}

\textbf{Purpose:} build shared understanding before group work.

\textbf{Topics to revisit:}
\begin{itemize}
    \item Why naïve pattern search is slow (Slides 4--6).
    \item Boyer--Moore logic: bad-character + good-suffix skipping (Slides 9--12).
    \item Why pattern preprocessing alone is insufficient for large-scale searching.
    \item Why indexing the \emph{text} (genome) matters (Slides 17--18).
\end{itemize}

\textbf{Guiding Questions:}
\begin{itemize}
    \item ``How do Boyer--Moore's rules avoid redundant comparisons?''
    \item ``Why is the pattern scanned from right to left in Boyer--Moore?''
    \item ``Why is text indexing crucial for read mapping and multi-query workloads?''
\end{itemize}

\begin{solution}
Stress the shift from ``searching faster'' (Boyer--Moore) to ``preprocessing for repeated queries'' (indexing).  
Students should clearly connect the motivation for suffix arrays and FM-index to the impracticality of repeated naïve search. 
\end{solution}

% % ----------------------------------------------
% \section*{20--40 min -- Mini Case / Exercises Block}

% \textbf{Format:} Students work in pairs or small groups; instructor circulates.

% \textbf{Activities:}
% \begin{enumerate}
%     \item \textbf{Repeat classification:} Given short descriptions or sequences, classify them as tandem/interspersed, autonomous/non-autonomous, or RNA vs.\ DNA intermediate (Slides 12--20).
%     \item \textbf{Mechanism matching:} Match each TE with its propagation mechanism (TPRT, cut-and-paste, LTR retrotransposition; Slides 14--21).
%     \item \textbf{Repeat detection reasoning:} Given simplified outputs from TRF, RECON, or RepeatScout, decide which tool best fits which repeat type and why (Slides 33--39, 46--53).
%     \item \textbf{Evolution scenario:} Evaluate how TE activation in a stressed population might affect genome size or generate structural variants (Slides 25--28).
% \end{enumerate}

% \begin{solution}
% Encourage students to refer to specific slide visuals during discussion. Emphasize mechanism recognition: LINE = TPRT; LTR = retroviral-like; DNA transposons = TIRs + cut/paste. For detection tools, TRF → tandem repeats; RECON/RepeatScout → TE families; RED → rapid genome-wide detection.
% \end{solution}

\section*{20--38 min — Structured Small-Group Discussion (Rotating Roles)}

Students form groups of 3.  
Roles rotate every 6--7 minutes.

\subsection*{Roles}
\begin{itemize}
    \item \textbf{Summarizer:} explains how $k$-mer tables and hash tables provide fast fixed-length searches (Slides 19--23, 28--29).
    \item \textbf{Questioner:} asks about tradeoffs in suffix trees, suffix arrays, and FM-index (Slides 32--52).
    \item \textbf{Connector:} links indexing structures to real tools (mappers, aligners, search engines).
\end{itemize}

\subsection*{Starter Question}
\begin{quote}
Why do we need different indexing structures for fixed-length and variable-length pattern searches?
\end{quote}

\begin{solution}
Expected points:
\begin{itemize}
    \item $k$-mer tables excel for fixed-length lookup but fail for variable-length queries (Slides 19--21).
    \item Suffix trees support arbitrary-length prefix queries but require huge memory (Slides 33--35).
    \item Suffix arrays provide efficient lexicographic search with smaller space (Slides 36--38).
    \item FM-index enables compressed and fast backward search (Slides 44--52).
\end{itemize}
Encourage comparison of memory vs.\ time tradeoffs.
\end{solution}


% ======================================================
\section*{40--50 min — Mini Applied Exercise Block}

See handout.

\begin{solution}
Prompt reasoning, not computation.  
Guide learners toward:
\begin{itemize}
    \item Understanding skip heuristics conceptually.
    \item Seeing $k$-mers as fast fixed-length seeds.
    \item Understanding how suffix arrays support prefix search.
    \item Grasping that BWT is reversible and basis for FM-index.
\end{itemize}
\end{solution}


% ----------------------------------------------
\section*{50--60 min -- Synthesis Discussion}

\textbf{Prompts:}
\begin{itemize}
    \item ``If you needed to index the human genome on a laptop, which structure would you choose and why?''
    \item ``How does the FM-index achieve both small size and fast lookup?''
    \item ``What characteristics of genomes make indexing harder than indexing typical English text?''
\end{itemize}

\begin{solution}
Key memory comparisons from slides:
\begin{itemize}
    \item Suffix tree: $>$45 GB for human genome (Slide 35).
    \item Suffix array: $\sim$12 GB (Slide 37).
    \item FM-index: $\sim$1.5 GB (Slide 52).
\end{itemize}
Push students toward articulating why compression and rank/select operations achieve speed + small footprint.
\end{solution}



\end{document}