\documentclass[11pt,a4paper]{article}

% ===============================
% Packages
% ===============================
\usepackage[margin=2.2cm]{geometry}
\usepackage{setspace}
\usepackage{titlesec}
\usepackage{enumerate}
\usepackage{tabularx}
\usepackage{booktabs}
\usepackage{amsmath}
\usepackage{hyperref}
\usepackage{fancyhdr}
\usepackage{parskip}
\usepackage{lmodern}
\usepackage{xcolor}
\usepackage{environ}

% ===============================
% Toggle solutions (instructor notes)
% ===============================
\newif\ifsolutions
\solutionsfalse    % <-- Student version
%\solutionstrue      % <-- Instructor version

\definecolor{instrblue}{RGB}{0,70,160}

\NewEnviron{solution}{%
  \ifsolutions
    \par\medskip
    \noindent\textbf{\color{instrblue}Instructor note. }\color{instrblue}\BODY
    \par\medskip\color{black}
  \fi
}

\newcommand{\instrnote}[1]{\ifsolutions{\color{instrblue}\emph{[#1]}}\fi}

% ===============================
% Header / footer
% ===============================
\pagestyle{fancy}
\fancyhf{}
\lhead{BINF301 — Discussion Plan}
\rhead{90-Min Session}
\rfoot{\thepage}
\ifsolutions
  \lhead{Instructor Version — Notes Included}
\fi

% Section spacing
\titlespacing*{\section}{0pt}{10pt}{4pt}
\titlespacing*{\subsection}{0pt}{7pt}{3pt}

% ===============================
\begin{document}

\begin{center}
    {\LARGE \textbf{90-Minute Discussion Session Plan}}\\[4pt]
    {\Large \textbf{Lecture 2: Sequencing \& k-mers}}\\[10pt]
    \textbf{Course:} BINF301 — Computational Biology\\
    \textbf{Instructor:} [Enter Name] \\
    \textbf{Date:} [Enter Date]
\end{center}

\vspace{0.5cm}

% ==========================================================
\section*{0-10 min — Warm-Up}

\begin{itemize}
    \item Icebreaker: Which sequencing technology (Sanger, Illumina, PacBio, Nanopore) are you most familiar with?
    \item Starter prompt: What surprised you most about differences between sequencing generations?
\end{itemize}

\begin{solution}
Goal: lower anxiety barriers, establish conversational readiness, surface prior knowledge, and encourage all students to speak once very early.
\end{solution}

% ==========================================================
\section*{10-25 min — Think-Pair-Share}

\textbf{Prompt:}  
Based on the lecture slides, what do you think is the main limitation of each sequencing technology (1st, 2nd, 3rd generation)?

\textbf{Follow-up:}  
How do quality scores influence downstream analysis?

\begin{solution}
Students should identify:  
• Sanger: low throughput.  
• Illumina: short reads, drop in 3' quality.  
• PacBio/Nanopore: higher error rates (raw), cost.  
Quality scores → trimming → fewer spurious k-mers.
\end{solution}

% ==========================================================
\section*{25-45 min — Structured Group Discussion}

\textbf{Roles:}
\begin{itemize}
    \item \textbf{Summarizer:} Recap FASTA vs FASTQ formats and Phred quality scores.
    \item \textbf{Questioner:} Bring one question about basecalling or sequencing error models.
    \item \textbf{Connector:} Link k-mers to sequencing output and quality issues.
\end{itemize}

\textbf{Starter question:}  
Why do sequencing errors inflate the number of unique k-mers?

\begin{solution}
Errors → new erroneous sequences → appear once → left-side tail of k-mer frequency plot → overestimates genome size.
\end{solution}

% ==========================================================
\section*{\textit{Break (15 min)}}

\medskip

% ==========================================================
\section*{45-65 min — Applied Exercises Block}

Students work collaboratively on structured exercises, see handout.

% % ------------------------------
% \subsection*{Exercise 1 — Compare Sequencing Technologies}

% Using the sequencing table, choose the best platform for:
% \begin{enumerate}[(a)]
%     \item genome assembly
%     \item variant calling
%     \item metagenomics
% \end{enumerate}

% \begin{solution}
% (a) PacBio HiFi or Nanopore; (b) Illumina for SNPs, HiFi for SVs; (c) Illumina (deep metagenomics) or Nanopore (long-read species resolution).
% \end{solution}

% % ------------------------------
% \subsection*{Exercise 2 — Quality Score Interpretation}

% Given FASTQ quality string:
% \begin{verbatim}
% @@@DDDDFFFFFGHIJ
% \end{verbatim}

% Tasks:
% \begin{itemize}
%     \item Convert characters to Phred scores.
%     \item Identify low-quality regions.
%     \item Discuss trimming impact on k-mer spectra.
% \end{itemize}

% \begin{solution}
% Mapping: @→Q31, D→Q35, F→Q37, etc. Values are high; no low-quality tail here. In real data, trimming prevents spurious unique k-mers.
% \end{solution}

% % ------------------------------
% \subsection*{Exercise 3 — k-mer Counting Thought Experiment}

% Sequence:
% \begin{verbatim}
% ATGATGCT
% \end{verbatim}

% Tasks:
% \begin{itemize}
%     \item List all 4-mers.
%     \item Count frequencies.
%     \item Predict effects of error or repeat.
% \end{itemize}

% \begin{solution}
% 4-mers: ATGA, TGAT, GATG, ATGC, TGCT (all count=1).  
% Errors → new unique singletons.  
% Repeats → higher-frequency duplicates.
% \end{solution}

% % ------------------------------
% \subsection*{Exercise 4 — Mini Case: Genome Size Estimation}

% Toy k-mer dataset:
% \begin{center}
% \begin{tabular}{l c}
% \texttt{ATG} & 20 \\
% \texttt{TGA} & 19 \\
% \texttt{GAT} & 22 \\
% \texttt{ATC} & 1 \\
% \texttt{TCA} & 1 \\
% \end{tabular}
% \end{center}

% Tasks:
% \begin{itemize}
%     \item Identify likely sequencing errors.
%     \item Estimate approximate coverage.
%     \item Discuss how GenomeScope models errors, repeats, heterozygosity.
% \end{itemize}

% \begin{solution}
% Singletons (ATC, TCA) = errors.  
% Main peak ~20×.  
% GenomeScope fits mixture model:  
% errors → low-frequency tail;  
% repeats → 2C, 3C peaks;  
% heterozygosity → sub-peak at C/2.
% \end{solution}

% % ==========================================================
% \section*{65-80 min — Open Forum \& Optional Deep Dive}

% \begin{itemize}
%     \item Students ask questions inspired by exercises.
%     \item Optional: revisit Exercise 4 to connect Poisson model to coverage estimation.
% \end{itemize}

% \begin{solution}
% Encourage students to compare empirical k-mer spectra to ideal Poisson predictions.
% \end{solution}

% ==========================================================
\section*{80-90 min — Reflection \& Wrap-Up}

\begin{itemize}
    \item One takeaway from today?
    \item One question to revisit next time?
\end{itemize}

\begin{solution}
Useful to collect feedback for pacing, confusion points, and future topics.
\end{solution}

% ==========================================================
\end{document}