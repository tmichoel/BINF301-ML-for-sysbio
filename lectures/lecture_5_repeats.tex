\documentclass[11pt,a4paper]{article}

% ===============================
% Packages
% ===============================
\usepackage[margin=2.2cm]{geometry}
\usepackage{titlesec}
\usepackage{parskip}
\usepackage{lmodern}
\usepackage{fancyhdr}
\usepackage{xcolor}
\usepackage{environ}
\usepackage{enumerate}
\usepackage{hyperref}

% ===============================
% Toggle for instructor notes
% ===============================
\newif\ifsolutions
\solutionsfalse   % <-- Student version
%\solutionstrue     % <-- Instructor version

\definecolor{instrblue}{RGB}{0,70,160}

\NewEnviron{solution}{%
  \ifsolutions
    \par\medskip
    \noindent\textbf{\color{instrblue}Instructor note. }\color{instrblue}\BODY
    \par\medskip\color{black}
  \fi
}

% ===============================
% Header / footer
% ===============================
\pagestyle{fancy}
\fancyhf{}
\lhead{BINF301 -- Discussion Plan}
\rhead{Lecture 3: Genome Assembly (60 min)}
\rfoot{\thepage}
\ifsolutions
  \lhead{Instructor Version -- Notes Included}
\fi

% Section spacing
\titlespacing*{\section}{0pt}{10pt}{4pt}
\titlespacing*{\subsection}{0pt}{7pt}{3pt}

% ===============================
\begin{document}

\begin{center}
    {\LARGE \textbf{60-Minute Discussion Session Plan}}\\[4pt]
    {\Large \textbf{Lecture 5: Repeats}}\\[10pt]
    \textbf{Course:} BINF301 -- Computational Biology\\
    \textbf{Instructor:} Tom Michoel\\
    \textbf{Date:} 28/01/2026\\
    \textbf{Created with Copilot}
\end{center}

\vspace{0.5cm}

% ==========================================================
\section*{0-5 min -- Warm-Up}

%\textbf{Purpose:} Reduce anxiety barriers and encourage early participation.


\textbf{Warm-up prompt:}

`Did any of the genome-size figures or TE diagrams stand out or raise questions?''


\begin{solution}
Goal: get students comfortable speaking early; activate recall from key ideas on Slides 3--6 and 10--12 (C-value variation, what counts as repetitive DNA). Encourage brief answers rather than deep explanations at this stage.
\end{solution}

% ----------------------------------------------
\section*{5--20 min -- Guided Concept Walkthrough}

\textbf{Purpose:} Establish shared conceptual grounding before deeper discussion.

\textbf{Topics:}
\begin{itemize}
    \item The C-value paradox and genome-size variability (Slides 3--6).
    \item Types of repetitive DNA: tandem vs.\ interspersed, autonomy, replication mode (Slides 10--12).
    \item Transposable elements: LTR, LINE, SINE, DNA transposons (Slides 14--22).
    \item Genome evolution impacts: structural variation, plasticity, HTT (Slides 25--28).
    \item Why repeat detection is difficult (degeneration, partial copies, divergent families; Slides 32--39).
\end{itemize}

\textbf{Guiding prompts:}
\begin{itemize}
    \item ``Why can genome size vary so widely between similar organisms?''
    \item ``How would you explain the difference between LINEs and SINEs to a beginner?''
    \item ``What features help us identify whether a TE is autonomous?''
    \item ``Why do degenerate repeats cause trouble for computational tools?''
\end{itemize}

\begin{solution}
Keep the walkthrough brisk. Anchor students by pointing to major diagrams: e.g., C-value plots, TE life-cycle schematics, classification tables, and the TE propagation mechanisms on Slides 17--21. Clarify autonomy vs.\ non-autonomy and replication intermediates.
\end{solution}

% % ----------------------------------------------
% \section*{20--40 min -- Mini Case / Exercises Block}

% \textbf{Format:} Students work in pairs or small groups; instructor circulates.

% \textbf{Activities:}
% \begin{enumerate}
%     \item \textbf{Repeat classification:} Given short descriptions or sequences, classify them as tandem/interspersed, autonomous/non-autonomous, or RNA vs.\ DNA intermediate (Slides 12--20).
%     \item \textbf{Mechanism matching:} Match each TE with its propagation mechanism (TPRT, cut-and-paste, LTR retrotransposition; Slides 14--21).
%     \item \textbf{Repeat detection reasoning:} Given simplified outputs from TRF, RECON, or RepeatScout, decide which tool best fits which repeat type and why (Slides 33--39, 46--53).
%     \item \textbf{Evolution scenario:} Evaluate how TE activation in a stressed population might affect genome size or generate structural variants (Slides 25--28).
% \end{enumerate}

% \begin{solution}
% Encourage students to refer to specific slide visuals during discussion. Emphasize mechanism recognition: LINE = TPRT; LTR = retroviral-like; DNA transposons = TIRs + cut/paste. For detection tools, TRF → tandem repeats; RECON/RepeatScout → TE families; RED → rapid genome-wide detection.
% \end{solution}

% ----------------------------------------------
\section*{20--40 min -- Open Reflection \& Deep Dive}

\textbf{Discussion prompts:}
\begin{itemize}
    \item ``Which classification dimension (mechanism, structure, autonomy) is hardest for you?''
    \item ``Why does repeat degeneration complicate family classification?''
    \item ``How do LINEs and SINEs co-evolve? Why do SINEs depend on LINE machinery?''
    \item ``Which detection approach (alignment-based, \(k\)-mer–based, HMM-based) best handles highly degenerate repeats?''
    \item ``How do repeats affect genome assembly, annotation, and error-prone regions?''
\end{itemize}

\begin{solution}
Use this block to unify biological and computational angles. Encourage reasoning rather than fact recall. For example: degeneration → lower sequence identity → clustering ambiguity; TE co-evolution → shared machinery but different autonomy; HMMs (RED) handle weak signals better than strict \(k\)-mer frequency.
\end{solution}

% ----------------------------------------------
\section*{40--55 min -- Assignment}

Work on the Portfolio Assignment Genomics, task data retrieval.

% ----------------------------------------------
\section*{55--60 min -- Wrap-Up}

\textbf{Closing prompt:}

``What remains confusing about how repeat detection tools differentiate families?''


\begin{solution}
Collect a quick diagnostic of students' comprehension. These answers help target the next lecture (e.g., more depth on TE evolution, or practical lab work on RepeatMasker/RepeatModeler).
\end{solution}

% ----------------------------------------------

\end{document}