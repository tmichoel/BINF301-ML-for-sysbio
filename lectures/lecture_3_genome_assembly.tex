\documentclass[11pt,a4paper]{article}

% ===============================
% Packages
% ===============================
\usepackage[margin=2.2cm]{geometry}
\usepackage{titlesec}
\usepackage{parskip}
\usepackage{lmodern}
\usepackage{fancyhdr}
\usepackage{xcolor}
\usepackage{environ}
\usepackage{enumerate}
\usepackage{hyperref}

% ===============================
% Toggle for instructor notes
% ===============================
\newif\ifsolutions
\solutionsfalse   % <-- Student version
%\solutionstrue     % <-- Instructor version

\definecolor{instrblue}{RGB}{0,70,160}

\NewEnviron{solution}{%
  \ifsolutions
    \par\medskip
    \noindent\textbf{\color{instrblue}Instructor note. }\color{instrblue}\BODY
    \par\medskip\color{black}
  \fi
}

% ===============================
% Header / footer
% ===============================
\pagestyle{fancy}
\fancyhf{}
\lhead{BINF301 — Discussion Plan}
\rhead{Lecture 3: Genome Assembly (60 min)}
\rfoot{\thepage}
\ifsolutions
  \lhead{Instructor Version — Notes Included}
\fi

% Section spacing
\titlespacing*{\section}{0pt}{10pt}{4pt}
\titlespacing*{\subsection}{0pt}{7pt}{3pt}

% ===============================
\begin{document}

\begin{center}
    {\LARGE \textbf{60-Minute Discussion Session Plan}}\\[4pt]
    {\Large \textbf{Lecture 3: Genome Assembly}}\\[10pt]
    \textbf{Course:} BINF301 — Computational Biology\\
    \textbf{Instructor:} Tom Michoel\\
    \textbf{Date:} 21/01/2026\\
    \textbf{Created with Copilot}
\end{center}

\vspace{0.5cm}

% ==========================================================
\section*{0-8 min — Warm-Up}

%\textbf{Purpose:} Reduce anxiety barriers and encourage early participation.

\textbf{Warm-up prompts:}
\begin{itemize}
    \item ``Which part of genome assembly feels the most intuitive to you: OLC, De Bruijn graphs, or error correction?''
    \item ``What slide image from the pre-read stood out as confusing or interesting?''
\end{itemize}

\begin{solution}
Use slides 3-4 (overview and ``three laws of assembly'') to connect warm-up responses to main themes.
\end{solution}

% ==========================================================
\section*{8-20 min — Guided Concept Walkthrough}

\textbf{Purpose:} Establish a shared understanding before deeper work.

\textbf{Walkthrough topics:}
\begin{itemize}
    \item What genome assembly is (Slide 3).
    \item Two algorithms: Overlap-Layout-Consensus (Slides 5-12) vs.\ De Bruijn Graphs (Slides 13-25).
    \item Why repeats cause problems (Slide 4).
\end{itemize}

\textbf{Guiding prompts:}
\begin{itemize}
    \item ``How would you describe the main difference between an OLC graph and a De Bruijn graph?''
    \item ``Why does OLC look for a Hamiltonian path, while De Bruijn graphs use an Eulerian path?''
    \item ``What happens when sequencing is not perfect?'' (Slide 18)
\end{itemize}

\begin{solution}
Ensure students understand: OLC focuses on reads as nodes (Hamiltonian), DBG focuses on \emph{k}-mers as edges (Eulerian). Mention Hamiltonian path is NP-hard, Eulerian path is linear-time.
\end{solution}

% ==========================================================
\section*{20-40 min — Mini Case / Exercises Block}

Students work collaboratively on structured exercises, see handout.


% ==========================================================
\section*{40-55 min — Open Reflection \& Deep Dive}

Prompts:
\begin{itemize}
    \item ``Which concept was hardest today: overlaps, repeat resolution, Eulerian vs.\ Hamiltonian paths?''
    \item ``What causes tangles in De Bruijn graphs? Biological repeats? Sequencing errors?''
    \item ``Why are error-correction methods (Slides 27-38) essential before assembly?''
\end{itemize}

\begin{solution}
Encourage connections: repeat-induced branching, error-induced tips/bubbles, and how correction tools simplify the graph.
\end{solution}

% ==========================================================
\section*{55-60 min — Wrap-Up}

Prompts:
\begin{itemize}
    \item ``One key insight from today?''
    \item ``Which algorithm (OLC vs DBG) feels more intuitive to you and why?''
    \item ``What should we revisit next lecture?''
\end{itemize}

\begin{solution}
Use answers to adjust pacing and revisit weak spots at the beginning of next session.
\end{solution}

% ==========================================================
\end{document}