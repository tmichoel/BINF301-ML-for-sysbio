\documentclass[11pt,a4paper]{article}

% ===============================
% Packages
% ===============================
\usepackage[margin=2.2cm]{geometry}
\usepackage{setspace}
\usepackage{titlesec}
\usepackage{enumerate}
\usepackage{tabularx}
\usepackage{booktabs}
\usepackage{amsmath}
\usepackage{hyperref}
\usepackage{fancyhdr}
\usepackage{parskip}
\usepackage{lmodern}
\usepackage{xcolor}
\usepackage{environ}

% ===============================
% Toggle solutions (instructor notes)
% ===============================
\newif\ifsolutions
\solutionsfalse    % <-- Student version
%\solutionstrue      % <-- Instructor version

\definecolor{instrblue}{RGB}{0,70,160}

\NewEnviron{solution}{%
  \ifsolutions
    \par\medskip
    \noindent\textbf{\color{instrblue}Instructor note. }\color{instrblue}\BODY
    \par\medskip\color{black}
  \fi
}

\newcommand{\instrnote}[1]{\ifsolutions{\color{instrblue}\emph{[#1]}}\fi}

% ===============================
% Header / footer
% ===============================
\pagestyle{fancy}
\fancyhf{}
\lhead{BINF301 -- Discussion Plan}
\rhead{90-Min Session}
\rfoot{\thepage}
\ifsolutions
  \lhead{Instructor Version -- Notes Included}
\fi

% Section spacing
\titlespacing*{\section}{0pt}{10pt}{4pt}
\titlespacing*{\subsection}{0pt}{7pt}{3pt}

% ===============================
\begin{document}

\begin{center}
    {\LARGE \textbf{90-Minute Discussion Session Plan}}\\[4pt]
    {\Large \textbf{Lecture 6: Genome Annotation}}\\[10pt]
    \textbf{Course:} BINF301 -- Computational Biology\\
    \textbf{Instructor:} Tom Michoel \\
    \textbf{Date:} 2/2/2026\\
    \textbf{Created with Copilot}
\end{center}

\vspace{0.5cm}

\section*{0--10 min — Warm-Up}

\textbf{Prompt:}

\begin{quote}
    Which part of genome annotation seems most challenging: repeat masking, gene prediction, or functional annotation?
\end{quote}



\begin{solution}
Goal: activate prior knowledge and reduce anxiety around annotation workflows.  
Use this time to challenge the misconception that annotation is only about coding genes.  
(Slides 2–3: definition of genome annotation)
\end{solution}

% ---------------------------------------------------------
\section*{10--25 min — Think–Pair–Share}

\textbf{Main Prompt:}
\begin{quote}
Why is repeat masking the first step in most genome annotation pipelines?
\end{quote}

\textbf{Follow-up Questions:}
\begin{itemize}
    \item ``What specific problems do unmasked repeats cause for ab initio gene predictors?''
    \item ``How does softmasking vs.\ hardmasking influence downstream tools?''
    \item ``Why can repeat detection be slow, and why might RED be used instead of RepeatMasker?''
\end{itemize}

\begin{solution}
Expected points:  
\begin{itemize}
    \item Repeats create spurious ORFs, inflating false positives in gene prediction (Slides 5–8).  
    \item Masking reduces search space and improves accuracy of HMM-based predictors.  
    \item RepeatModeler + RepeatMasker are thorough but slow; RED is fast but not family-aware.  
    \item Students should articulate differences between softmasking (“lowercase”) and hardmasking (“Ns”).  
\end{itemize}
\end{solution}

% ---------------------------------------------------------
\section*{25--45 min — Structured Group Discussion}

Students form groups of three with rotating roles.

\subsection*{Starter Question}
\begin{quote}
Why is eukaryotic gene prediction dramatically harder than prokaryotic annotation?
\end{quote}

\subsection*{Roles}
\begin{itemize}
    \item \textbf{Summarizer:} Explain differences between prokaryotic and eukaryotic gene structure (Slides 10–14 vs.\ 20–24).
    \item \textbf{Questioner:} Raise a question about ab initio gene prediction (HMM states, GC content, training).  
    \item \textbf{Connector:} Link RNA-seq/protein homology evidence to gene prediction quality (Slides 25–27).
\end{itemize}



\begin{solution}
Expected discussion points:
\begin{itemize}
    \item Prokaryotes: no introns, high gene density → ORF detection is straightforward (Slides 12–14).  
    \item Eukaryotes: introns, splice sites, long intergenic regions, alternative splicing → HMMs + external evidence needed (Slides 20–24).  
    \item Students should compare Prokka (modular, prokaryotic) vs.\ BRAKER/Augustus (complex, model-based).  
\end{itemize}
\end{solution}

\section*{\emph{Break (15 min)}}

\medskip
% ---------------------------------------------------------

\section*{45--60 min — Deep Dive: Integrating Evidence Sources}

\textbf{Discussion Prompts:}
\begin{itemize}
    \item ``When is ab initio prediction alone insufficient?''
    \item ``How do BRAKER2 and BRAKER3 integrate intrinsic HMMs with extrinsic RNA-seq and protein hints?''
    \item ``What are limitations of homology-based predictions in non-model organisms?''
    \item ``How would you resolve conflicts between RNA-seq evidence and ab initio predictions?''
\end{itemize}

\begin{solution}
Emphasize:
\begin{itemize}
    \item BRAKER pipelines unify HMM predictions with RNA-seq or homology hints (Slides 27–30).  
    \item Annotation often requires weighing conflicting data.  
    \item Students should appreciate the interdependence of evidence sources: HMMs give structure, RNA-seq gives boundaries, homology gives function.  
\end{itemize}
\end{solution}

% ---------------------------------------------------------

\section*{60--70 min — Deep Dive:  Annotation Quality Assessment}

\textbf{Prompts:}
\begin{itemize}
    \item ``What exactly does BUSCO completeness mean?''
    \item ``How does OMArk detect contamination and taxonomic inconsistency?''
    \item ``Why might duplicated BUSCOs indicate fragmentation rather than real gene duplication?''
\end{itemize}

\begin{solution}
Key notes to guide the discussion:
\begin{itemize}
    \item BUSCO evaluates presence/absence of conserved lineage-specific orthologs (Slides 41–43).  
    \item OMArk evaluates phylogenetic consistency using OMA gene families and k-mer mapping (Slides 45–52).  
    \item Multiple copies of BUSCO genes may indicate misannotation, fragmentation, or partial duplicates rather than true biological paralogs.  
\end{itemize}
\end{solution}


% ---------------------------------------------------------
\section*{70--90 min — Assignment}

Students work on the Portfolio Assignment Genomics.

% Exercises are conceptual, discussion-focused, no computation needed.

% \subsection*{Exercise 1 — Effects of Missing Repeat Masking}
% Provide a short sequence containing LINE/SINE-like repeats.  
% Students identify where ab initio predictors might falsely detect ORFs.

% \subsection*{Exercise 2 — HMM State Reasoning}
% Show a simplified exon–intron structure diagram.  
% Students list which HMM states must be traversed: initial exon, intron donor, intron acceptor, internal exon, terminal exon, intergenic.  
% (Slides 21–22)

% \subsection*{Exercise 3 — RNA-seq Evidence}
% Provide a small RNA-seq splice-junction schematic.  
% Students infer exon boundaries and explain why splice-aware alignment is needed.  
% (Slide 25)

% \subsection*{Exercise 4 — Function Annotation Reasoning}
% Give examples of proteins with Pfam domain hits vs.\ BLAST hits.  
% Students decide which annotation is more reliable and why.  
% (Slides 32–39)

% \begin{solution}
% Instructor reminders:
% \begin{itemize}
%     \item Reinforce why periodicity (3n) modeling is essential for coding potential (Slide 22).  
%     \item Emphasize RNA-seq’s limitations: low-expression genes and tissue specificity.  
%     \item Clarify that domain-based and homology-based functional annotation are probabilistic.  
% \end{itemize}
% \end{solution}



% % ---------------------------------------------------------
% \section*{85--90 min — Reflection \& Wrap-Up}

% \textbf{Closing prompts:}
% \begin{itemize}
%     \item ``What misconception about genome annotation was corrected today?''
%     \item ``Which step—repeat masking, gene prediction, functional assignment, or quality assessment—still confuses you?''
%     \item ``Which tool (RepeatMasker, Prokka, Augustus, BUSCO) would you like demonstrated next session?''
% \end{itemize}

% \begin{solution}
% Use responses to calibrate next lectures.  
% Students often request demos of Augustus training, RNA-seq hint generation, or BUSCO runs.
% \end{solution}


\end{document}