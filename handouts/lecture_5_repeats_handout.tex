\documentclass[11pt,a4paper]{article}

% ===============================
% Packages
% ===============================
\usepackage[margin=2.2cm]{geometry}
\usepackage{setspace}
\usepackage{titlesec}
\usepackage{enumerate}
\usepackage{tabularx}
\usepackage{booktabs}
\usepackage{amsmath}
\usepackage{hyperref}
\usepackage{fancyhdr}
\usepackage{parskip}
\usepackage{lmodern}
\usepackage{xcolor}
\usepackage{environ}
\usepackage{verbatim}
\usepackage{tikz}
\usetikzlibrary{arrows.meta,positioning,calc,decorations.pathmorphing}

% ===============================
% Toggle: student vs instructor
% ===============================
\newif\ifsolutions
\solutionsfalse    % <-- Student version (no solutions)
%\solutionstrue       % <-- Instructor version (solutions visible)

% ===============================
% Styles (Style 1 — Clean Minimalist)
% ===============================
\definecolor{instrblue}{RGB}{0,70,160}
\definecolor{edgeblue}{RGB}{30,90,180}

% Diagram styles
\tikzset{
  read/.style={rectangle, rounded corners=2pt, draw=black!60, fill=black!5, inner sep=3pt, minimum width=18pt, minimum height=12pt, font=\footnotesize},
  node/.style={circle, draw=black!60, fill=black!5, inner sep=1.8pt, minimum size=16pt, font=\footnotesize},
  edge/.style={-Latex, very thin, draw=edgeblue},
  altedge/.style={-Latex, very thin, draw=black!60, dashed},
  pathhl/.style={-Latex, semithick, draw=black}, % highlight path (monochrome friendly)
  err/.style={-Latex, thin, draw=black, dashed}, % shows erroneous or low-confidence edges
  tip/.style={-Latex, thin, draw=black!60, densely dotted}
}

% ===============================
% Robust solution environment
% ===============================
\NewEnviron{solution}{%
  \ifsolutions
    \par\medskip
    \noindent\textbf{\color{instrblue}Solution. }\color{instrblue}\BODY
    \par\medskip\color{black}
  \fi
}

\newcommand{\instrnote}[1]{\ifsolutions{\color{instrblue}\emph{[#1]}}\fi}

% ===============================
% Header / Footer
% ===============================
\pagestyle{fancy}
\fancyhf{}
\lhead{Sequencing \& Assembly — Lecture 3}
\rhead{BINF301}
\rfoot{\thepage}
\ifsolutions
  \lhead{Instructor Version — Solutions Included}
\fi

% Section spacing
\titlespacing*{\section}{0pt}{10pt}{4pt}
\titlespacing*{\subsection}{0pt}{7pt}{3pt}

% ===============================
\begin{document}

\begin{center}
    {\LARGE \textbf{Lecture 5: Repeats}}\\[6pt]
    {\Large \textbf{Student Handout \& In-Class Exercises}}\\[10pt]
    \textbf{Course:} BINF301 — Computational Biology \\
    \textbf{Instructor:} Tom Michoel \\
    \textbf{Date:} 28/01/2026\\
    \textbf{Created with Copilot}
\end{center}

\vspace{0.5cm}

\section{The C-value Paradox}
\textbf{C-value:} the amount of DNA in a haploid genome (pg). Eukaryotic genome size does \emph{not} correlate with organismal ``complexity''; closely related species can have vastly different genome sizes. Much of the size variation is explained by non-coding and repetitive DNA. % Slides: BINF301_Lecture5_Repeats.pdf
\begin{solution}
Key messages to stress:
(1) Genome size varies over orders of magnitude among eukaryotes.
(2) Differences are largely driven by repeat/TE content rather than gene number.
(3) Avoid equating genome size with organismal complexity.
\end{solution}

\section{Types of Repetitive DNA}
Repeat classification (non-exhaustive):
\begin{itemize}
  \item \textbf{By location:} interspersed vs.\ tandem; segmental duplications.
  \item \textbf{By structure:} simple/low-complexity vs.\ composite elements (e.g., LTR).
  \item \textbf{By autonomy:} autonomous vs.\ non-autonomous (``hitch-hiking'').
  \item \textbf{By replication mode:} copy-and-paste (retrotransposons) vs.\ cut-and-paste (DNA transposons).
  \item \textbf{By activity:} active vs.\ inactivated relics.
\end{itemize}
\begin{solution}
Helpful framing: \emph{Interspersed} repeats are dispersed copies (often TEs), while \emph{tandem} repeats are adjacent copies (e.g., satellites). Autonomy indicates whether the element encodes the machinery needed for its own mobilization.
\end{solution}

\section{Simple (Tandem) Repeats}
\begin{itemize}
  \item \textbf{Satellites:} short motifs (e.g., \texttt{TATA...}) repeated head-to-tail; common in centromeres.
  \item \textbf{Telomeres:} vertebrate consensus \texttt{TTAGGG}.
  \item \textbf{Mechanism:} often formed by \emph{slipped-strand mispairing} during replication.
\end{itemize}
\begin{solution}
Emphasize the distinction between tandem repeats (periodic, adjacent) and interspersed elements (TEs). Simple repeats are important for chromosome structure (centromeres/telomeres), but can also complicate assembly.
\end{solution}

\section{Transposable Elements (TEs)}
TEs are mobile DNA first described by McClintock. Two broad classes:
\subsection{Retrotransposons (copy-and-paste)}
\begin{itemize}
  \item \textbf{LTR retrotransposons:} share features with retroviruses (but lack viral envelope/capsid).
  \item \textbf{Non-LTR:} \textbf{LINEs} (autonomous; \(\sim\)17\% of human genome, mostly inactive); \textbf{SINEs} (non-autonomous; up to \(\sim\)15\%).
  \item \textbf{Mechanism:} RNA intermediate, reverse transcription (e.g., TPRT for LINE-1).
\end{itemize}
\subsection{DNA Transposons (cut-and-paste)}
\begin{itemize}
  \item Typically have \textbf{TIRs} (terminal inverted repeats) and generate \textbf{TSDs} at insertion.
  \item Families include Type II transposons, Helitrons; some encode their own transposase; Polintons (Type III) encode polymerase/integrase and have TIRs.
\end{itemize}
\begin{solution}
Talking points:
(1) \textbf{Autonomous vs.\ non-autonomous:} LINEs carry ORF1/ORF2; SINEs \emph{lack} these and parasitize LINE machinery.
(2) \textbf{Mechanistic fingerprints:} TIRs/TSDs for DNA transposons; LTRs for LTR retrotransposons; target-primed reverse transcription for LINE-1.
(3) \textbf{Activity:} In humans, most copies are ancient and inactive, but remnants dominate genome composition.
\end{solution}

\section{Repeats and Genome Evolution}
\begin{itemize}
  \item TE proliferation expands genomes; inactive copies accumulate.
  \item Repeats can generate structural variation, gene duplication/deletion, new regulatory elements, and occasionally new genes.
  \item \textbf{Horizontal Transposon Transfer (HTT):} transfer across species via vectors (e.g., parasites/viruses); detection remains challenging.
\end{itemize}
\begin{solution}
Balance the narrative: although many TEs are neutral or deleterious at insertion, over long timescales they contribute to innovation (cis-regulatory rewiring, exon shuffling) and plasticity. HTT is rare in eukaryotes but increasingly documented.
\end{solution}

\section{Repeat Detection: Core Tools and Ideas}
\subsection{TRF (Tandem Repeat Finder)}
Sliding-window search for periodic \(k\)-mers; identifies candidate regions, checks periodic spacing/consistency, and reports tandem repeat units.
\begin{solution}
Students should know that TRF is specialized for tandem repeats (satellites, microsatellites) and not for interspersed TEs.
\end{solution}

\subsection{RECON}
Alignment-based \emph{de novo} TE discovery:
\begin{enumerate}
  \item Compute pairwise alignments; define \emph{images} (aligned subsequences).
  \item Cluster images via single-linkage (syntopy rules) to infer \emph{elements}.
  \item Build inter-element graph; refine edges (primary/secondary) and detect triangles to avoid false family merges; connected components define \emph{families}.
\end{enumerate}
\begin{solution}
Contrast with TRF: RECON is for interspersed repeats/TEs; handles partial copies/degeneration via graph refinement.
\end{solution}

\subsection{RepeatScout}
Seed-and-extend on frequent \(k\)-mers:
\begin{itemize}
  \item Start from high-frequency \(k\)-mers; greedily extend to maximize a consensus scoring function (with penalties for dangling ends).
  \item Iteratively extract consensus families and remove their instances from the \(k\)-mer table.
\end{itemize}
\begin{solution}
RepeatScout tends to be faster than all-vs-all alignment (RECON) and is effective when families still share abundant \(k\)-mers.
\end{solution}

\subsection{RED (REpeat Detector)}
Signal-processing + HMM approach:
\begin{itemize}
  \item Assign adjusted \(k\)-mer frequency scores per base; smooth, find local maxima; delineate candidate repeat vs.\ non-repeat regions.
  \item Train an HMM on candidates, then scan genome to label repeats.
\end{itemize}
\begin{solution}
Key selling points: very fast, self-learning, detects both tandem and interspersed repeats; but does not classify families by type.
\end{solution}

\subsection{RepeatModeler \& RepeatMasker}
\begin{itemize}
  \item \textbf{RepeatModeler}: orchestrates multiple \emph{de novo} tools to build a species-specific repeat library.
  \item \textbf{RepeatMasker}: screens a genome with known/\emph{de novo} libraries to mask repeats (often very time-consuming).
\end{itemize}
\begin{solution}
A typical pipeline: \emph{RepeatModeler} to build library \(\rightarrow\) \emph{RepeatMasker} to annotate/mask. Useful for downstream gene prediction and variant calling.
\end{solution}

\newpage
% --------------------------------------------------------------------
\section{Exercises}

\subsection{Exercise 1 — C-Value Paradox Reasoning}
\textbf{Prompt.} The lungfish genome is \(>\)100\(\times\) larger than the human genome. Explain why genome size can increase dramatically without increasing organismal complexity.

\begin{solution}
Large genomes typically reflect accumulation of repeats/TEs, polyploidy events, and reduced DNA removal mechanisms, not expanded protein-coding gene sets. Many TE insertions become inactive but persist, inflating genome size while contributing little to gene count or ``complexity.''
\end{solution}

\subsection{Exercise 2 — Classifying Repeat Types}
\textbf{Prompt.} Classify each as \emph{tandem / interspersed / autonomous / non-autonomous}:
\begin{enumerate}
  \item Vertebrate telomere \texttt{TTAGGG}
  \item SINE (e.g., Alu)
  \item LINE-1
  \item Centromeric satellites
\end{enumerate}

\begin{solution}
1) Telomere TTAGGG: \emph{tandem} (simple repeat). \quad
2) SINE/Alu: \emph{interspersed}, \emph{non-autonomous} (uses LINE machinery). \quad
3) LINE-1: \emph{interspersed}, \emph{autonomous} (encodes ORF1/ORF2). \quad
4) Centromeric satellites: \emph{tandem} repeats.
\end{solution}

\subsection{Exercise 3 — TE Replication Mechanisms}
\textbf{Prompt.} Match TE type to mechanism:
\begin{center}
\begin{tabular}{ll}
A.\ DNA transposons & 1.\ Copy-and-paste (RNA intermediate) \\
B.\ LINEs & 2.\ Cut-and-paste (DNA intermediate) \\
C.\ LTR retrotransposons & 3.\ Target-primed reverse transcription \\
\end{tabular}
\end{center}

\begin{solution}
A\(\rightarrow\)2 (DNA transposons cut-and-paste); \quad
B\(\rightarrow\)3 (LINE-1 uses TPRT); \quad
C\(\rightarrow\)1 (LTR retrotransposons copy-and-paste via RNA).
\end{solution}

\subsection{Exercise 4 — Genome Evolution by TEs}
\textbf{Prompt.} Provide one beneficial and one harmful evolutionary consequence of TE activity.

\begin{solution}
\textbf{Beneficial:} Creation of novel regulatory elements (TE-derived enhancers/promoters), exon shuffling, or substrate for gene duplication. \\
\textbf{Harmful:} Disruption of coding/regulatory regions upon insertion; ectopic recombination between repeats causing deletions/rearrangements.
\end{solution}

\subsection{Exercise 5 — Using TRF}
\textbf{Prompt.} For sequence \texttt{ATGATGATGATGCCCGTA}:
\begin{enumerate}
  \item Identify the tandem repeat motif.
  \item How many copies?
\end{enumerate}

\begin{solution}
(a) Motif = \texttt{ATG}. \quad
(b) There are 4 adjacent copies: \texttt{ATG ATG ATG ATG} followed by non-repetitive sequence.
\end{solution}

\subsection{Exercise 6 — RECON vs.\ RepeatScout}
\textbf{Prompt.} Explain how RECON and RepeatScout differ in discovering repeat families.

\begin{solution}
\textbf{RECON:} all-vs-all alignments to cluster \emph{images} \(\rightarrow\) infer \emph{elements} \(\rightarrow\) inter-element graph refinement \(\rightarrow\) families; robust to partial/degenerate copies but computationally heavier. \\
\textbf{RepeatScout:} frequent \(k\)-mer seeding with greedy consensus extension; faster and relies on abundance of short exact words; may struggle if repeats are too diverged to yield frequent \(k\)-mers.
\end{solution}

\subsection{Exercise 7 — TE Inactivation}
\textbf{Prompt.} Give two reasons why most LINEs are inactive in humans.

\begin{solution}
(1) Accumulation of disabling mutations in ORF1/ORF2 or promoter regions over evolutionary time; \\
(2) Host defense mechanisms (DNA methylation, small RNAs) suppress TE expression; \\
(3) Truncation at insertion (common for LINE-1) yields defective copies.
\end{solution}

\subsection{Exercise 8 — RED in Practice (Conceptual)}
\textbf{Prompt.} Why does RED adjust \(k\)-mer frequencies and then train an HMM?

\begin{solution}
Adjusted \(k\)-mer scores highlight sequence segments with unexpected repetitiveness beyond base composition; smoothing and local maxima detection define candidate ``repeat'' vs.\ ``non-repeat'' regions. The HMM then generalizes these patterns to label the whole genome efficiently and consistently.
\end{solution}


\end{document}
% --------------------------------------------------------------------
\section{Quick Checklist for Students}
\begin{itemize}
  \item Distinguish tandem vs.\ interspersed repeats and identify common examples.
  \item Explain the key mechanisms of LTR, LINE, SINE, and DNA transposons.
  \item Describe at least two ways TEs impact genome evolution.
  \item Know when to use TRF vs.\ \emph{de novo} TE tools (RECON/RepeatScout/RED).
  \item Understand why RepeatModeler+RepeatMasker are standard for genome-wide masking.
\end{itemize}
