\documentclass[11pt,a4paper]{article}

% ===============================
% Packages
% ===============================
\usepackage[margin=2.2cm]{geometry}
\usepackage{setspace}
\usepackage{titlesec}
\usepackage{enumerate}
\usepackage{tabularx}
\usepackage{booktabs}
\usepackage{amsmath}
\usepackage{hyperref}
\usepackage{fancyhdr}
\usepackage{parskip}
\usepackage{lmodern}
\usepackage{xcolor}
\usepackage{environ}
\usepackage{verbatim}
\usepackage{tikz}
\usetikzlibrary{arrows.meta,positioning,calc,decorations.pathmorphing}

% ===============================
% Toggle: student vs instructor
% ===============================
\newif\ifsolutions
\solutionsfalse    % <-- Student version (no solutions)
%\solutionstrue       % <-- Instructor version (solutions visible)

% ===============================
% Styles (Style 1 — Clean Minimalist)
% ===============================
\definecolor{instrblue}{RGB}{0,70,160}
\definecolor{edgeblue}{RGB}{30,90,180}

% Diagram styles
\tikzset{
  read/.style={rectangle, rounded corners=2pt, draw=black!60, fill=black!5, inner sep=3pt, minimum width=18pt, minimum height=12pt, font=\footnotesize},
  node/.style={circle, draw=black!60, fill=black!5, inner sep=1.8pt, minimum size=16pt, font=\footnotesize},
  edge/.style={-Latex, very thin, draw=edgeblue},
  altedge/.style={-Latex, very thin, draw=black!60, dashed},
  pathhl/.style={-Latex, semithick, draw=black}, % highlight path (monochrome friendly)
  err/.style={-Latex, thin, draw=black, dashed}, % shows erroneous or low-confidence edges
  tip/.style={-Latex, thin, draw=black!60, densely dotted}
}

% ===============================
% Robust solution environment
% ===============================
\NewEnviron{solution}{%
  \ifsolutions
    \par\medskip
    \noindent\textbf{\color{instrblue}Solution. }\color{instrblue}\BODY
    \par\medskip\color{black}
  \fi
}

\newcommand{\instrnote}[1]{\ifsolutions{\color{instrblue}\emph{[#1]}}\fi}

% ===============================
% Header / Footer
% ===============================
\pagestyle{fancy}
\fancyhf{}
\lhead{Sequencing \& Assembly — Lecture 3}
\rhead{BINF301}
\rfoot{\thepage}
\ifsolutions
  \lhead{Instructor Version — Solutions Included}
\fi

% Section spacing
\titlespacing*{\section}{0pt}{10pt}{4pt}
\titlespacing*{\subsection}{0pt}{7pt}{3pt}

% ===============================
\begin{document}

\begin{center}
    {\LARGE \textbf{Lecture 9: Bioinformatics Tools \& Databases}}\\[6pt]
    {\Large \textbf{Student Handout \& In-Class Exercises}}\\[10pt]
    \textbf{Course:} BINF301 — Computational Biology \\
    \textbf{Instructor:} Tom Michoel \\
    \textbf{Date:} 11/2/2026\\
    \textbf{Created with Copilot}
\end{center}

\vspace{0.5cm}

% =========================================================
\section{ELIXIR \& Core Data Resources (Slides 3--7)}

ELIXIR is a Europe-wide infrastructure that coordinates and develops:
\begin{itemize}
    \item Life-science data resources,
    \item computing platforms,
    \item standardized metadata and interoperability tools,
    \item training and community networks.
\end{itemize}


ELIXIR promotes the FAIR principles:
\begin{itemize}
    \item \textbf{Findable:} indexed, searchable, identifiable;
    \item \textbf{Accessible:} open protocols and metadata access;
    \item \textbf{Interoperable:} shared vocabularies, structured data;
    \item \textbf{Reusable:} rich metadata, clear licensing, provenance.
\end{itemize}


\begin{solution}
Make clear to students that ELIXIR is about \emph{both} data and tools, and that FAIR principles guide modern research data management. 
\end{solution}

% =========================================================
\section{What is a Bioinformatic Tool? (Slides 9--12)}

A bioinformatic tool is software that:
\begin{itemize}
    \item implements one or more computational algorithms,
    \item accepts biological data as input,
    \item produces processed/analyzed/annotated biological output,
    \item may have various interfaces (CLI, GUI, web, API).
\end{itemize}


Metadata describing tools (format expectations, algorithms, dependencies) is essential for reproducibility.

\begin{solution}
Students often think ``tools = big software packages.'' Instead, stress that even a small script can be a tool if it implements a meaningful computational function.
\end{solution}

% =========================================================
\section{Finding \& Using Tools (Slides 13--17)}

Researchers typically discover tools through:
\begin{itemize}
    \item publications,
    \item community recommendations,
    \item conferences,
    \item tool registries such as \textbf{bio.tools},
    \item the Galaxy tools ecosystem.
\end{itemize}


\subsection{bio.tools Registry (Slides 14--18)}
\begin{itemize}
    \item ELIXIR-funded tool registry,
    \item 29,227 entries (Feb.\ 2024),
    \item uses unique identifiers and ontology-based tags.
\end{itemize}


\begin{solution}
Highlight key teaching point: metadata-rich registries like bio.tools enable better discovery, evaluation, and reuse of computational tools.
\end{solution}

% =========================================================
\section{Installing \& Running Tools (Slides 19--22)}

Bioinformatics tools are typically installed via:
\begin{itemize}
    \item Linux/Unix native binaries,
    \item macOS (sometimes compatible),
    \item Windows via WSL,
    \item Conda/Bioconda/Mamba for dependency management,
    \item Docker or Singularity containers for full reproducibility,
    \item GitHub repositories and manual compilation.
\end{itemize}


\begin{solution}
Students should understand the tradeoffs:  
Conda = easy dependency resolution;  
Containers = fully reproducible environment;  
Manual builds = flexible but error-prone.
\end{solution}

% =========================================================
\section{Bioinformatic Workflows (Slides 24--28)}

Large analyses combine multiple steps → workflows.

\subsection{Workflow Languages (Slide 25)}
\begin{itemize}
    \item \textbf{CWL}: command-line focused.
    \item \textbf{Snakemake}: Python-based; hybrid scripting + rules.
    \item \textbf{Nextflow}: scalable, portable, used by NF-core.
\end{itemize}


\subsection{Workflow Databases (Slide 28)}
\begin{itemize}
    \item \textbf{WorkflowHub}: FAIR-compliant, domain-agnostic repository.
    \item \textbf{NF-core}: curated Nextflow pipelines.
\end{itemize}


\begin{solution}
Explain reproducibility: workflows encode both tool calls and metadata, enabling robust, shareable analyses.
\end{solution}

% =========================================================
\section{Databases (Slides 31--45)}

A true database is:
\begin{itemize}
    \item structured, machine-readable,
    \item built on a database management system (e.g.\ SQL),
    \item consistent and queryable.
\end{itemize}


\subsection{Data Deposition (Slide 33)}
Many journals require deposition in:
\begin{itemize}
    \item \textbf{SRA} (NCBI),
    \item \textbf{ENA} (Europe),
    \item \textbf{DDBJ} (Japan),
\end{itemize}
which synchronize globally.


\subsection{Widely Used Databases (Slides 38--45)}
\begin{itemize}
    \item \textbf{ENA, GenBank, SRA} — sequence archives
    \item \textbf{UniProt} — protein knowledgebase
    \item \textbf{Ensembl} — annotated genomes
    \item \textbf{OrthoDB} — ortholog groups
    \item \textbf{BRENDA} — enzyme function
    \item \textbf{KEGG} — pathways
    \item \textbf{PDB} — protein structures
    \item \textbf{InterPro/PFAM} — protein domains
\end{itemize}


\begin{solution}
Point out that each database solves different biological questions; knowing which one to query is a key skill in bioinformatics.
\end{solution}

% =========================================================
\section{Programmatic Access (Slides 47--55)}

Modern datasets support machine-readable interfaces:
\begin{itemize}
    \item \textbf{REST APIs} for structured HTTP output (JSON, XML),
    \item \textbf{BioMart} for data mining,
    \item \textbf{NCBI E-utils} for scripted access (ESearch, ESummary, EFetch),
    \item \textbf{NCBI Datasets} for bulk genome/gene downloads.
\end{itemize}


\begin{solution}
Instructor should clarify the role of REST: all parameters in URL, standardized output, ideal for scripting and automation.
\end{solution}

% =========================================================
\section{Exercises (with Solutions)}

\subsection{Exercise 1 — FAIR Principles}

Match:
\begin{itemize}
    \item[a.] Dataset includes rich metadata + license.
    \item[b.] Database provides JSON API.
    \item[c.] Sequence has stable accession number.
    \item[d.] Workflow uses controlled vocabularies.
\end{itemize}

with: Findable, Accessible, Interoperable, Reusable.

\begin{solution}
a→Reusable;  
b→Accessible;  
c→Findable;  
d→Interoperable.  
\end{solution}

% ---------------------------------------------------------
\subsection{Exercise 2 — bio.tools Exploration}

Pick a known tool and identify:
\begin{itemize}
    \item tool name,
    \item bio.tools identifier,
    \item input type,
    \item output type.
\end{itemize}

\begin{solution}
Example: \textbf{FastQC}.  
Inputs: FASTQ files.  
Outputs: HTML reports + per-base quality statistics.  
bio.tools identifier varies (conceptual example).
\end{solution}

% ---------------------------------------------------------
\subsection{Exercise 3 — Tool Installation Methods}

Choose best installation method:
\begin{enumerate}
    \item Tool with many dependencies all available in Conda.  
    \item Tool requiring an old Linux library.  
    \item Complex workflow with many collaborators.  
    \item Simple script with no dependencies.
\end{enumerate}

Options: Conda, container, workflow language, manual install.

\begin{solution}
1→Conda;  
2→Container (ensures environment consistency);  
3→Nextflow/Snakemake workflow;  
4→Manual install (lightweight).
\end{solution}

% ---------------------------------------------------------
\subsection{Exercise 4 — Database Choice}

Match task → database:
\begin{itemize}
    \item Retrieve all human protein sequences.
    \item Find 3D kinase structure.
    \item Download metagenome reads.
    \item Retrieve orthologs across insects.
\end{itemize}

\begin{solution}
Human proteins → UniProt;  
Kinase structure → PDB;  
Metagenome reads → SRA/ENA;  
Orthologs → OrthoDB.
\end{solution}

% ---------------------------------------------------------
\subsection{Exercise 5 — REST vs E-utils}

Explain the difference.

\begin{solution}
REST: generic web API using structured URLs, returns JSON/XML, widely used across databases.  
E-utils: NCBI-specific REST-like API with ESearch/EFetch operations for querying and retrieving NCBI data.
\end{solution}

% ---------------------------------------------------------
\subsection{Exercise 6 — Workflow Languages True/False}

\begin{enumerate}
    \item CWL focuses on command-line tools.  
    \item Snakemake is Bash-based.  
    \item Nextflow emphasizes scalability.  
    \item NF-core workflows use Galaxy.
\end{enumerate}

\begin{solution}
1: True.  
2: False (Python-based).  
3: True.  
4: False (Nextflow-only).  
\end{solution}

% =========================================================

\end{document}