\documentclass[11pt,a4paper]{article}

% ===============================
% Packages
% ===============================
\usepackage[margin=2.2cm]{geometry}
\usepackage{setspace}
\usepackage{titlesec}
\usepackage{enumerate}
\usepackage{tabularx}
\usepackage{booktabs}
\usepackage{amsmath}
\usepackage{hyperref}
\usepackage{fancyhdr}
\usepackage{parskip}
\usepackage{lmodern}
\usepackage{xcolor}
\usepackage{environ}
\usepackage{verbatim}
\usepackage{tikz}
\usetikzlibrary{arrows.meta,positioning,calc,decorations.pathmorphing}

% ===============================
% Toggle: student vs instructor
% ===============================
\newif\ifsolutions
\solutionsfalse    % <-- Student version (no solutions)
%\solutionstrue       % <-- Instructor version (solutions visible)

% ===============================
% Styles (Style 1 — Clean Minimalist)
% ===============================
\definecolor{instrblue}{RGB}{0,70,160}
\definecolor{edgeblue}{RGB}{30,90,180}

% Diagram styles
\tikzset{
  read/.style={rectangle, rounded corners=2pt, draw=black!60, fill=black!5, inner sep=3pt, minimum width=18pt, minimum height=12pt, font=\footnotesize},
  node/.style={circle, draw=black!60, fill=black!5, inner sep=1.8pt, minimum size=16pt, font=\footnotesize},
  edge/.style={-Latex, very thin, draw=edgeblue},
  altedge/.style={-Latex, very thin, draw=black!60, dashed},
  pathhl/.style={-Latex, semithick, draw=black}, % highlight path (monochrome friendly)
  err/.style={-Latex, thin, draw=black, dashed}, % shows erroneous or low-confidence edges
  tip/.style={-Latex, thin, draw=black!60, densely dotted}
}

% ===============================
% Robust solution environment
% ===============================
\NewEnviron{solution}{%
  \ifsolutions
    \par\medskip
    \noindent\textbf{\color{instrblue}Solution. }\color{instrblue}\BODY
    \par\medskip\color{black}
  \fi
}

\newcommand{\instrnote}[1]{\ifsolutions{\color{instrblue}\emph{[#1]}}\fi}

% ===============================
% Header / Footer
% ===============================
\pagestyle{fancy}
\fancyhf{}
\lhead{Sequencing \& Assembly — Lecture 3}
\rhead{BINF301}
\rfoot{\thepage}
\ifsolutions
  \lhead{Instructor Version — Solutions Included}
\fi

% Section spacing
\titlespacing*{\section}{0pt}{10pt}{4pt}
\titlespacing*{\subsection}{0pt}{7pt}{3pt}

% ===============================
\begin{document}

\begin{center}
    {\LARGE \textbf{Lecture 8: Functional and Comparative Genomics}}\\[6pt]
    {\Large \textbf{Student Handout \& In-Class Exercises}}\\[10pt]
    \textbf{Course:} BINF301 — Computational Biology \\
    \textbf{Instructor:} Tom Michoel \\
    \textbf{Date:} 9/2/2026\\
    \textbf{Created with Copilot}
\end{center}

\vspace{0.5cm}


% =========================================================
\section{What is Genomic "Function"? (Slides 3--5)}

There is no single agreed-upon definition of ``function'' in genomic sequences.
Lecture 8 describes three complementary frameworks:

\subsection{Genetic Approach}
\begin{itemize}
    \item Function inferred from \textbf{loss-of-function phenotypes} after mutation or interference.
    \item Limited by redundancy and subtle or context-specific phenotypes.
\end{itemize}


\subsection{Evolutionary Approach}
\begin{itemize}
    \item \textbf{Comparative genomics}: conserved sequences often under purifying selection.
    \item Works well for protein-coding regions; regulatory regions evolve rapidly.
\end{itemize}


\subsection{Biochemical Approach (ENCODE)}
\begin{itemize}
    \item Biochemical assays (ChIP-seq, DNase-seq, ATAC-seq, etc.) identify active or bound regions.
    \item Biochemical signal does not always imply causal function.
\end{itemize}


\begin{solution}
Important teaching point: students should recognize that ``function'' varies by experimental approach—each approach captures a different biological concept (phenotype, constraint, biochemical activity).
\end{solution}

% =========================================================
\section{The ENCODE Project (Slides 7--12)}

ENCODE (Encyclopedia of DNA Elements) aims to catalog:
\begin{itemize}
    \item Coding genes
    \item Non-coding RNAs
    \item Regulatory elements
    \item Epigenetic marks and chromatin features
\end{itemize}
Data are available via the ENCODE portal and the UCSC Genome Browser.


\begin{solution}
ENCODE provides multi-assay functional data (RNA-seq, ChIP-seq, ATAC-seq, etc.). Students should understand differences in the biological interpretations of these data types.
\end{solution}

% =========================================================
\section{Functional Genomics Methods (Slides 14--22)}

\subsection{ChIP-seq (Slide 15)}
Chromatin Immunoprecipitation + sequencing detects \textbf{DNA-protein interactions}.  


\subsection{ATAC-seq (Slide 16)}
Assesses \textbf{chromatin accessibility} using transposase accessibility.  


\subsection{DNase-seq (Slide 17)}
Detects DNase I hypersensitive sites → open chromatin.  


\subsection{FAIRE-seq (Slide 18)}
Enriches nucleosome-depleted, regulatory regions.  


\subsection{RNA-seq (Slide 19)}
Measures transcription, gene expression, isoforms, and non-coding RNAs.  


\subsection{Ribo-seq (Slide 20)}
Profiles ribosome-protected RNA fragments → \textbf{translation activity}.  


\subsection{DNA Methylation Assays (Slide 21)}
Bisulfite conversion distinguishes methylated vs.\ unmethylated cytosines.  


\subsection{Functional Assays (Slide 22)}
CRISPR knockouts, RNAi, overexpression → link genomic regions to phenotypes.  


\begin{solution}
Highlight differences between assays:
\begin{itemize}
    \item Accessibility (ATAC/FAIRE/DNase) vs binding (ChIP) vs transcription (RNA-seq) vs translation (Ribo-seq).
    \item Students often conflate RNA-seq with ``functional'' evidence; clarify it measures expression, not biological effect.
\end{itemize}
\end{solution}

% =========================================================
\section{Comparative Genomics (Slides 24--41)}

\subsection{Whole-Genome Alignment (Slides 25--28)}
\begin{itemize}
    \item \textbf{Mauve}: multi-genome alignment with rearrangement awareness.
    \item \textbf{MUMmer}: fast pairwise alignment using maximal unique matches.
\end{itemize}


\subsection{Long-Read Genome Alignment (Slides 31--32)}
\textbf{Minimap2} uses a seed–chain–align procedure with minimizers for fast large-scale alignment.  


\subsection{Variant Detection (Slides 33--34)}
Read-to-reference alignment identifies SNPs and indels (e.g.\ GATK).  


\subsection{Orthology and Gene Comparisons (Slides 36--41)}
\begin{itemize}
    \item Reciprocal Best Hit (RBH) to detect orthologs.
    \item \textbf{Orthofinder}: finds orthogroups, builds gene trees, infers species trees, identifies duplications/losses.
\end{itemize}


\begin{solution}
Comparative genomics links molecular biology to evolution:
\begin{itemize}
    \item Conservation → functionality
    \item Breakpoints → structural evolution
    \item Orthogroups → shared ancestry vs lineage-specific changes
\end{itemize}
\end{solution}

% =========================================================
\section{Exercises (with Solutions)}

\subsection{Exercise 1 — Three Definitions of Function}
Give one advantage and one limitation for each approach: genetic, evolutionary, biochemical.

\begin{solution}
\textbf{Genetic:}  
Advantage: strong causal inference.  
Limitation: redundancy hides phenotypes; low throughput.

\textbf{Evolutionary:}  
Advantage: identifies constrained regions genome-wide.  
Limitation: fails for rapidly evolving regulatory regions.

\textbf{Biochemical:}  
Advantage: scalable; maps many element types.  
Limitation: biochemical signal may reflect correlation, not causation.
\end{solution}

% ---------------------------------------------------------
\subsection{Exercise 2 — Matching Assays to Functions}
Match:
\begin{enumerate}
    \item ChIP-seq  
    \item ATAC-seq  
    \item RNA-seq  
    \item Ribo-seq  
    \item Bisulfite sequencing  
\end{enumerate}
to:
\begin{itemize}
    \item[a.] Open chromatin  
    \item[b.] Transcription levels  
    \item[c.] DNA-protein binding  
    \item[d.] Translation activity  
    \item[e.] Cytosine methylation  
\end{itemize}

\begin{solution}
1→c,\; 2→a,\; 3→b,\; 4→d,\; 5→e.
\end{solution}

% ---------------------------------------------------------
\subsection{Exercise 3 — ENCODE Data Interpretation}
Choose one assay (e.g.\ ChIP-seq, ATAC-seq, RNA-seq).  
State:
\begin{itemize}
    \item What biological question it answers
    \item What type of data output it produces
\end{itemize}

\begin{solution}
Examples:  
\textbf{ATAC-seq:} reveals open chromatin; output = peaks representing accessible regions.  
\textbf{ChIP-seq:} detects protein-DNA binding; output = enrichment peaks.  
\textbf{RNA-seq:} measures expression; output = read counts / transcripts.  
\end{solution}

% ---------------------------------------------------------
\subsection{Exercise 4 — Comparative Genomics Tools}
For each tool, specify its best use case.

\begin{solution}
\textbf{Mauve:} multiple small genomes, rearrangement-aware alignment.  
\textbf{MUMmer:} pairwise genome alignment, very fast.  
\textbf{Minimap2:} long-read mapping, assembly-to-assembly alignment.  
\end{solution}

% ---------------------------------------------------------
\subsection{Exercise 5 — Orthology Reasoning}
A1's best hit is B1, and B1's best hit is A1.  
A2's best hit is B2, but B2's best hit is A1.

\begin{solution}
A1--B1: reciprocal best hit → likely orthologs.  
A2--B2: not reciprocal → suggests paralogy or gene family expansion; B2 may be more similar to duplicated A1.
\end{solution}

% ---------------------------------------------------------
\subsection{Exercise 6 — Variant Detection}
Reads at position: A, A, A, G, A, A.

Is this a SNP?

\begin{solution}
Yes: one read shows a G while reference = A.  
Confidence low: only 1/6 supports the alternative allele; could be sequencing error.  
True SNP calls typically require higher allele frequency and quality metrics.
\end{solution}

% =========================================================

\end{document}