\documentclass[11pt,a4paper]{article}

% ===============================
% Packages
% ===============================
\usepackage[margin=2.2cm]{geometry}
\usepackage{setspace}
\usepackage{titlesec}
\usepackage{enumerate}
\usepackage{tabularx}
\usepackage{booktabs}
\usepackage{amsmath}
\usepackage{hyperref}
\usepackage{fancyhdr}
\usepackage{parskip}
\usepackage{lmodern}
\usepackage{xcolor}
\usepackage{environ}
\usepackage{verbatim}
\usepackage{tikz}
\usetikzlibrary{arrows.meta,positioning,calc,decorations.pathmorphing}

% ===============================
% Toggle: student vs instructor
% ===============================
\newif\ifsolutions
\solutionsfalse    % <-- Student version (no solutions)
%\solutionstrue       % <-- Instructor version (solutions visible)

% ===============================
% Styles (Style 1 — Clean Minimalist)
% ===============================
\definecolor{instrblue}{RGB}{0,70,160}
\definecolor{edgeblue}{RGB}{30,90,180}

% Diagram styles
\tikzset{
  read/.style={rectangle, rounded corners=2pt, draw=black!60, fill=black!5, inner sep=3pt, minimum width=18pt, minimum height=12pt, font=\footnotesize},
  node/.style={circle, draw=black!60, fill=black!5, inner sep=1.8pt, minimum size=16pt, font=\footnotesize},
  edge/.style={-Latex, very thin, draw=edgeblue},
  altedge/.style={-Latex, very thin, draw=black!60, dashed},
  pathhl/.style={-Latex, semithick, draw=black}, % highlight path (monochrome friendly)
  err/.style={-Latex, thin, draw=black, dashed}, % shows erroneous or low-confidence edges
  tip/.style={-Latex, thin, draw=black!60, densely dotted}
}

% ===============================
% Robust solution environment
% ===============================
\NewEnviron{solution}{%
  \ifsolutions
    \par\medskip
    \noindent\textbf{\color{instrblue}Solution. }\color{instrblue}\BODY
    \par\medskip\color{black}
  \fi
}

\newcommand{\instrnote}[1]{\ifsolutions{\color{instrblue}\emph{[#1]}}\fi}

% ===============================
% Header / Footer
% ===============================
\pagestyle{fancy}
\fancyhf{}
\lhead{Sequencing \& Assembly — Lecture 3}
\rhead{BINF301}
\rfoot{\thepage}
\ifsolutions
  \lhead{Instructor Version — Solutions Included}
\fi

% Section spacing
\titlespacing*{\section}{0pt}{10pt}{4pt}
\titlespacing*{\subsection}{0pt}{7pt}{3pt}

% ===============================
\begin{document}

\begin{center}
    {\LARGE \textbf{Lecture 6: Genome Annotation}}\\[6pt]
    {\Large \textbf{Student Handout \& In-Class Exercises}}\\[10pt]
    \textbf{Course:} BINF301 — Computational Biology \\
    \textbf{Instructor:} Tom Michoel \\
    \textbf{Date:} 2/2/2026\\
    \textbf{Created with Copilot}
\end{center}

\vspace{0.5cm}

% ---------------------------------------------------------
\section{What is Genome Annotation? (Slides 2--3)}

A genome is a long nucleotide sequence; annotation aims to identify:
\begin{itemize}
    \item \textbf{Structural elements}: gene locations, repeats, ncRNAs, tRNAs, rRNAs, regulatory regions.
    \item \textbf{Functional elements}: functions assigned to predicted genes and RNAs.
\end{itemize}


\begin{solution}
Emphasize: annotation has two parts—finding features and assigning meaning.  
Make clear to students that much of a genome is \emph{noncoding} and must still be annotated (regulatory sites, ncRNAs, repeats).
\end{solution}

% ---------------------------------------------------------
\section{Annotation Workflow (Slide 3)}

A general annotation workflow includes:
\begin{enumerate}
    \item Repeat masking
    \item Gene prediction (ab initio, extrinsic, or combined)
    \item Prediction of additional functional elements (ncRNA, tRNA, rRNA)
    \item Functional annotation (domains, homology)
\end{enumerate}


\begin{solution}
Clarify that these steps occur in both prokaryotic and eukaryotic annotation, but with major differences in complexity.
\end{solution}

% ---------------------------------------------------------
\section{Repeat Masking (Slides 5--8)}

Many eukaryotic genomes contain \textbf{25--50\%} repeats. Masking repeats prevents false gene predictions and reduces the candidate search space.  


\subsection{Soft vs. Hard Masking (Slide 7)}
\textbf{Soft masking:} bases converted to lowercase.  
\textbf{Hard masking:} repeat regions replaced with \texttt{N}.

Tools:
\begin{itemize}
    \item RepeatModeler + RepeatMasker (de novo + masking; slow)
    \item RED (fast, no classification)
\end{itemize}


\begin{solution}
Make students aware that different gene-finders expect one masking type or the other.  
Soft masking retains information and is preferred when splice-aware aligners are used.
\end{solution}

% ---------------------------------------------------------
\section{Prokaryotic Genome Annotation (Slides 10--14)}

Prokaryotic genomes are simpler due to:
\begin{itemize}
    \item No introns
    \item High gene density
\end{itemize}


\subsection{Prokka Pipeline (Slide 11)}
Includes: Prodigal (CDS), RNAmmer (rRNA), Aragorn (tRNA), SignalP (signal peptides), Infernal (ncRNAs).  


\subsection{Prodigal (Slides 12--13)}
Uses:
\begin{itemize}
    \item ORF discovery with GC-bias scoring
    \item Dynamic programming to select best non-overlapping ORFs
    \item Hexamer frequencies to refine predictions
    \item RBS detection to refine start sites
\end{itemize}


\begin{solution}
Focus on: prokaryotic gene finding is essentially a classification of ORFs using statistics; no splice-site modeling needed.
\end{solution}

% ---------------------------------------------------------
\section{Noncoding RNA Detection (Slides 15--18)}

Tools:
\begin{itemize}
    \item RNAmmer (rRNA; HMMs)
    \item tRNAscan-SE (tRNA; covariance models)
    \item Infernal (general ncRNAs; covariance models)
\end{itemize}


\begin{solution}
Stress that RNA structure is essential—covariance models capture paired bases and secondary structure not available to simple HMMs.
\end{solution}

% ---------------------------------------------------------
\section{Eukaryotic Gene Prediction (Slides 20--24)}

Eukaryotic gene prediction is more complex due to introns, exon variation, UTRs, alternative splicing, and long intergenic regions.  


\subsection{Ab Initio Prediction (Slides 21--24)}
HMM-based models include states for:
\begin{itemize}
    \item Initial, internal, and terminal exons
    \item Introns (with splice sites)
    \item Intergenic regions
    \item Single-exon genes
\end{itemize}
Tools: \textbf{GeneMark}, \textbf{Augustus}.  


\begin{solution}
Explain training: unsupervised (GeneMark) vs.\ supervised (Augustus).  
Important: proper training greatly increases accuracy.
\end{solution}

% ---------------------------------------------------------
\section{Extrinsic Evidence (Slides 25--27)}

Sources:
\begin{itemize}
    \item RNA-seq (expression profiles, intron/exon boundaries)
    \item Protein homology
\end{itemize}

Integrated pipelines:
\begin{itemize}
    \item BRAKER2 / BRAKER3
    \item TSEBRA
\end{itemize}


\begin{solution}
Correction: RNA-seq alone is insufficient for accurate annotation.  
Require splice-aware alignment + integration with ab initio predictions.
\end{solution}

% ---------------------------------------------------------
\section{Functional Annotation (Slides 32--39)}

Approaches:
\begin{itemize}
    \item Domain-based: Pfam, CDD
    \item Homology-based: Reciprocal Best Hit (RBH), BLAST
    \item Combined systems: InterProScan, eggNOG mapper
\end{itemize}


\begin{solution}
Caution: domain presence does not guarantee exact function; annotations are probabilistic.
\end{solution}

% ---------------------------------------------------------
\section{Annotation Quality Assessment (Slides 41--52)}

\subsection{BUSCO (Slides 41--43)}
Uses lineage-specific universal single-copy orthologs to evaluate:
\begin{itemize}
    \item Completeness (single-copy, duplicated)
    \item Missing genes
\end{itemize}


\subsection{OMArk (Slides 45--52)}
Evaluates:
\begin{itemize}
    \item Proteome completeness
    \item Phylogenetic consistency
    \item Contamination
\end{itemize}
Uses OMA gene families and k-mer–based mapping (OMAmer).  


\begin{solution}
Distinguish roles: BUSCO tests “is most of the conserved gene set present?”, OMArk tests “is the annotation evolutionarily consistent?”.
\end{solution}

\newpage
% ---------------------------------------------------------
\section{Exercises}

\subsection{Exercise 1: Why Mask Repeats? (Slides 5--8)}

\textbf{Question:} Why must repeats be masked before annotation?

\begin{solution}
Repeats cause spurious ORFs and inflate false-positive gene calls; masking reduces errors and computation.
\end{solution}

% ---
\subsection{Exercise 2: Soft vs Hard Masking (Slide 7)}

\textbf{Sequence:} \texttt{ACGTCGGatatatatatCGATGA}

\textbf{Question:} Which masking type is this? What would the other type look like?

\begin{solution}
Lowercase = soft-masked.  
Hard-masked form: \texttt{ACGTCGGNNNNNNNNNNCGATGA}.
\end{solution}

% ---
\subsection{Exercise 3: Prokaryotic Simplicity (Slides 10--11)}

\textbf{Question:} List two reasons prokaryotic annotation is easier.

\begin{solution}
No introns; high gene density; straightforward ORFs.
\end{solution}

% ---
\subsection{Exercise 4: Ab Initio vs Extrinsic (Slides 21--27)}

\textbf{Question:} How do ab initio and extrinsic prediction differ?

\begin{solution}
Ab initio uses only genomic sequence/HMMs; extrinsic uses RNA-seq/protein evidence to guide boundaries.
\end{solution}

% ---
\subsection{Exercise 5: Functional Annotation (Slides 32--39)}

\textbf{Question:} Name two functional annotation strategies.

\begin{solution}
Domain detection (Pfam/CDD) and homology-based inference (BLAST/RBH).
\end{solution}

% ---
\subsection{Exercise 6: BUSCO Interpretation (Slides 41--43)}

\textbf{Question:} What does 70\% BUSCO completeness imply?

\begin{solution}
Many conserved genes missing or fragmented → incomplete assembly/annotation.
\end{solution}

% ---
\subsection{Exercise 7: Tool Matching}

Match tools to functions:

a. RNAmmer \\
b. tRNAscan-SE \\
c. RepeatMasker \\
d. InterProScan

\textbf{Purposes:}  
1. Protein domain detection  
2. rRNA prediction  
3. Repeat masking  
4. tRNA prediction

\begin{solution}
a–2,\; b–4,\; c–3,\; d–1.
\end{solution}

% ---------------------------------------------------------

\end{document}