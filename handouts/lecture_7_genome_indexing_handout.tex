\documentclass[11pt,a4paper]{article}

% ===============================
% Packages
% ===============================
\usepackage[margin=2.2cm]{geometry}
\usepackage{setspace}
\usepackage{titlesec}
\usepackage{enumerate}
\usepackage{tabularx}
\usepackage{booktabs}
\usepackage{amsmath}
\usepackage{hyperref}
\usepackage{fancyhdr}
\usepackage{parskip}
\usepackage{lmodern}
\usepackage{xcolor}
\usepackage{environ}
\usepackage{verbatim}
\usepackage{tikz}
\usetikzlibrary{arrows.meta,positioning,calc,decorations.pathmorphing}

% ===============================
% Toggle: student vs instructor
% ===============================
\newif\ifsolutions
\solutionsfalse    % <-- Student version (no solutions)
%\solutionstrue       % <-- Instructor version (solutions visible)

% ===============================
% Styles (Style 1 — Clean Minimalist)
% ===============================
\definecolor{instrblue}{RGB}{0,70,160}
\definecolor{edgeblue}{RGB}{30,90,180}

% Diagram styles
\tikzset{
  read/.style={rectangle, rounded corners=2pt, draw=black!60, fill=black!5, inner sep=3pt, minimum width=18pt, minimum height=12pt, font=\footnotesize},
  node/.style={circle, draw=black!60, fill=black!5, inner sep=1.8pt, minimum size=16pt, font=\footnotesize},
  edge/.style={-Latex, very thin, draw=edgeblue},
  altedge/.style={-Latex, very thin, draw=black!60, dashed},
  pathhl/.style={-Latex, semithick, draw=black}, % highlight path (monochrome friendly)
  err/.style={-Latex, thin, draw=black, dashed}, % shows erroneous or low-confidence edges
  tip/.style={-Latex, thin, draw=black!60, densely dotted}
}

% ===============================
% Robust solution environment
% ===============================
\NewEnviron{solution}{%
  \ifsolutions
    \par\medskip
    \noindent\textbf{\color{instrblue}Solution. }\color{instrblue}\BODY
    \par\medskip\color{black}
  \fi
}

\newcommand{\instrnote}[1]{\ifsolutions{\color{instrblue}\emph{[#1]}}\fi}

% ===============================
% Header / Footer
% ===============================
\pagestyle{fancy}
\fancyhf{}
\lhead{Sequencing \& Assembly — Lecture 3}
\rhead{BINF301}
\rfoot{\thepage}
\ifsolutions
  \lhead{Instructor Version — Solutions Included}
\fi

% Section spacing
\titlespacing*{\section}{0pt}{10pt}{4pt}
\titlespacing*{\subsection}{0pt}{7pt}{3pt}

% ===============================
\begin{document}

\begin{center}
    {\LARGE \textbf{Lecture 7: Searching Genomes and Genome Indexing}}\\[6pt]
    {\Large \textbf{Student Handout \& In-Class Exercises}}\\[10pt]
    \textbf{Course:} BINF301 — Computational Biology \\
    \textbf{Instructor:} Tom Michoel \\
    \textbf{Date:} 4/2/2026\\
    \textbf{Created with Copilot}
\end{center}

\vspace{0.5cm}

% =========================================================
\section{Text Search Basics (Slides 3--6)}

Genome search is defined as:  
\textbf{Given a pattern $P$ of length $n$ and a text $T$ of length $m$, find all occurrences of $P$ in $T$.}

Examples:
\begin{itemize}
    \item grep-like text search
    \item text editors (Ctrl+F)
    \item mapping reads to a reference genome
\end{itemize}

\textbf{Naïve Approach (Slides 4--6)}: 
Try every alignment of $P$ in $T$.

\begin{itemize}
    \item Worst case: $O(mn)$  
    \item Best case: $O(m)$
\end{itemize}

\begin{solution}
This simple algorithm does not scale to large genomes and motivates the need for smarter search and indexing.
\end{solution}

% =========================================================
\section{Boyer--Moore Algorithm (Slides 8--15)}

A fast exact matching algorithm using:
\begin{itemize}
    \item \textbf{Bad character rule:} After mismatch, shift pattern so mismatch aligns with rightmost matching character.
    \item \textbf{Good suffix rule:} Reuse matched suffix to shift pattern optimally.
\end{itemize}

\begin{solution}
Students should understand that Boyer–Moore reduces unnecessary comparisons by learning from mismatches and matches during alignment.
\end{solution}

% =========================================================
\section{Genome Indexing (Slides 17--18)}

Indexing preprocesses the text $T$ to support fast lookups for many patterns.  
Essential for:
\begin{itemize}
    \item read mapping
    \item repeated queries on a reference genome
\end{itemize}

% =========================================================
\section{k-mer Tables (Slides 19--23)}

A basic genome index storing each $k$-mer and the positions where it appears.

Advantages:
\begin{itemize}
    \item fast lookup (especially via hashing)
\end{itemize}

Limitations:
\begin{itemize}
    \item best when pattern length equals $k$
    \item not suitable for all substring lengths
\end{itemize}

\begin{solution}
This is conceptually simple and forms the basis of many alignment-free or seed-based methods in read mapping.
\end{solution}

% =========================================================
\section{Hash Tables (Slides 28--29)}

Hash tables store key-value pairs with near-constant lookup time.

Used for:
\begin{itemize}
    \item $k$-mer count tables
    \item presence/absence queries
\end{itemize}

% =========================================================
\section{Suffix Trees and Suffix Arrays (Slides 32--38)}

\subsection{Suffix Trees}
Compressed tries of all suffixes.  
Fast but extremely memory-heavy (on the order of 15 bytes per base).

\subsection{Suffix Arrays}
A space-efficient alternative:
\begin{itemize}
    \item store sorted suffix positions
    \item support fast binary search
    \item require much less space
\end{itemize}

\begin{solution}
Suffix arrays form the foundation of many modern genome index structures.
\end{solution}

% =========================================================
\section{Burrows--Wheeler Transform (Slides 39--41)}

A reversible transform that groups identical characters together, making the text more compressible.

\section{FM-index (Slides 44--52)}

FM-index = BWT + rank/select + partial suffix array sampling.

Properties:
\begin{itemize}
    \item extremely low memory usage (about 0.5 bytes/character)
    \item supports fast backward search
\end{itemize}

\begin{solution}
LF-mapping is the key operation students must grasp: it reconstructs the ordering of characters during backward search.
\end{solution}

% =========================================================
\section{Hands-On Exercises with Solutions}

\subsection{Exercise 1 — Naïve Search Simulation (Slides 3--6)}

Text: \texttt{ATGATCATGAC}  
Pattern: \texttt{ATG}

\textbf{Task:} List match positions.

\begin{solution}
Matches at:
\begin{itemize}
    \item positions 1--3
    \item positions 7--9
\end{itemize}
(0-based: 0 and 6)
\end{solution}

% ---------------------------------------------------------
\subsection{Exercise 2 — Boyer--Moore Skip (Slides 9--12)}

Pattern: \texttt{ATCGA}  
Mismatch char: \texttt{T}

\textbf{Task:} Compute skip distance using the bad character rule.

\begin{solution}
Rightmost \texttt{T} in pattern is at position 2 (1-based).  
If mismatch occurs at position 4, skip = $4 - 2 = 2$.
\end{solution}

% ---------------------------------------------------------
\subsection{Exercise 3 — Build a 3-mer Table (Slides 19--21)}

Text: \texttt{GATATAGA}

\textbf{Task:} List all 3-mers and positions.

\begin{solution}
\begin{tabular}{ll}
\texttt{GAT} & 1 \\
\texttt{ATA} & 2, 4 \\
\texttt{TAT} & 3 \\
\texttt{TAG} & 5 \\
\texttt{AGA} & 6 \\
\end{tabular}
\end{solution}

% ---------------------------------------------------------
\subsection{Exercise 4 — Suffix Array Binary Search (Slides 36--38)}

Suffixes of \texttt{GCTA\$} in sorted order:  
\begin{tabular}{ll}
0 & \texttt{\$} \\
1 & \texttt{A\$} \\
2 & \texttt{CTA\$} \\
3 & \texttt{GCTA\$} \\
4 & \texttt{TA\$} \\
\end{tabular}

\textbf{Task:} Does ``TA'' occur?

\begin{solution}
Binary search:
\begin{itemize}
    \item mid=2 → \texttt{CTA\$}: continue right
    \item mid=3 → \texttt{GCTA\$}: continue right
    \item index=4 → \texttt{TA\$}: match
\end{itemize}
Yes, ``TA'' occurs.
\end{solution}

% ---------------------------------------------------------
\subsection{Exercise 5 — BWT Construction (Slides 39--41)}

Text: \texttt{GAT\$}

\textbf{Task:} Build BWT.

\begin{solution}
Rotations: GAT\$, AT\$G, T\$GA, \$GAT  
Sorted:
\begin{tabular}{l}
\$GAT \\
AT\$G \\
GAT\$ \\
T\$GA \\
\end{tabular}
BWT = \texttt{T G \$ A}.
\end{solution}

% ---------------------------------------------------------
\subsection{Exercise 6 — FM-index Backward Step (Slides 45--48)}

BWT L = \texttt{T G \$ A A}  
Pattern = \texttt{GA}

\textbf{Task:} Compute initial range for final character \texttt{A}.

\begin{solution}
Sorted F = \texttt{\$ A A G T}.  
\texttt{A} occupies indices 2--3 (1-based), so initial range is [2,3].
\end{solution}

\end{document}