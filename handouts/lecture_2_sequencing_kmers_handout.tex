
\documentclass[11pt,a4paper]{article}

% ===============================
% Packages
% ===============================
\usepackage[margin=2.2cm]{geometry}
\usepackage{setspace}
\usepackage{titlesec}
\usepackage{enumerate}
\usepackage{tabularx}
\usepackage{booktabs}
\usepackage{amsmath}
\usepackage{hyperref}
\usepackage{fancyhdr}
\usepackage{parskip}
\usepackage{lmodern}
\usepackage{xcolor}
\usepackage{environ} % <-- robust environment capture

% ===============================
% Toggle: show/hide solutions
% ===============================
\newif\ifsolutions
\solutionsfalse   % <-- Student version (no solutions)
%\solutionstrue     % <-- Instructor version (solutions visible)

% ===============================
% Styling for instructor notes/solutions
% ===============================
\definecolor{instrblue}{RGB}{0,70,160}

% Robust solution environment:
% - Captures body as \BODY (from environ)
% - Prints only if \solutionstrue
\NewEnviron{solution}{%
  \ifsolutions
    \par\medskip
    \noindent\textbf{\color{instrblue}Solution. }\color{instrblue}\BODY
    \par\medskip\color{black}%
  \fi
}

% Optional inline instructor note (prints only in instructor mode)
\newcommand{\instrnote}[1]{\ifsolutions{\color{instrblue}\emph{[Instructor note: }#1\emph{]}}\fi}

% ===============================
% Header/Footer
% ===============================
\pagestyle{fancy}
\fancyhf{}
\lhead{Sequencing \& k-mers}
\rhead{BINF301 -- Lecture 2}
\rfoot{\thepage}
\ifsolutions
  \lhead{Instructor Version: Sequencing \& k-mers}
  \rhead{Solutions Included}
\fi

% Section spacing
\titlespacing*{\section}{0pt}{10pt}{4pt}
\titlespacing*{\subsection}{0pt}{7pt}{3pt}

\begin{document}

\begin{center}
    {\LARGE \textbf{Lecture 2: Sequencing \& k-mers}}\\[6pt]
    {\Large \textbf{Student Handout \& In-Class Exercises}}\\[10pt]
    \textbf{Course:} BINF301 --- Computational Biology \\
    \textbf{Instructor:} Tom Michoel \\
    \textbf{Date:} 19/01/2026\\
    \textbf{Created with Copilot}
\end{center}

\vspace{0.5cm}

% ----------------------------------------------------------
\section*{1. Overview}
This handout summarizes the central ideas from Lecture 2, including:
\begin{itemize}
    \item genome sequencing technologies,
    \item FASTA \& FASTQ formats,
    \item quality scores and read trimming,
    \item definition and use of \emph{k}-mers,
    \item genome size estimation using \emph{k}-mer distributions.
\end{itemize}

% ----------------------------------------------------------
\section*{2. Sequencing Technologies}
Below is a comparison of the four major sequencing platforms:

\begin{table}[h!]
\centering
\begin{tabularx}{\textwidth}{lccclX}
\toprule
\textbf{Technology} & \textbf{Read Length} & \textbf{Accuracy} & \textbf{Throughput} & \textbf{Notes} \\
\midrule
Sanger & 500–900 bp & Very high & Low & Chain termination;\\
& & & & used for small fragments. \\
Illumina & 100–300 bp & Very high & Very high & Short reads;\\
& & & & quality decreases at 3' end. \\
PacBio HiFi & 10–25 kb & Very high & Medium & Long reads;\\
& & & & extremely accurate HiFi mode. \\
Nanopore & 10 kb–1 Mb & Moderate & High & Electrical-signal based;\\
& & & &ultra-long reads; portable devices. \\
\bottomrule
\end{tabularx}
\end{table}


\textbf{Discussion prompts:}
\begin{itemize}
    \item Which technology is best for assembly, variant calling, or metagenomics?
    \item How do read length and accuracy interact?
\end{itemize}

% ----------------------------------------------------------
\section*{3. FASTA \& FASTQ}
\subsection*{FASTA}
\begin{itemize}
    \item Stores only sequences (DNA, RNA, protein).
    \item Simple two-line format: header + sequence.
\end{itemize}

\subsection*{FASTQ}
\begin{itemize}
    \item Stores nucleotide sequences \emph{and} per-base quality scores.
    \item Uses Phred encoding: \( Q = -10 \log_{10}(P_{\text{error}}) \).
    \item ASCII + 33 encoding for quality symbols.
\end{itemize}

% ----------------------------------------------------------
\section*{4. Quality Control \& Trimming}
\begin{itemize}
    \item Read quality often drops toward the 3' end of short reads.
    \item Adapters may be present and must be trimmed.
    \item Trimming improves downstream assembly and \emph{k}-mer profiling.
\end{itemize}

% ----------------------------------------------------------
\section*{5. k-mers}
A \emph{k}-mer is a substring of length \(k\) extracted from a sequence.  
\begin{itemize}
    \item Unique \emph{k}-mer count approximates genome size.
    \item Sequencing errors introduce low-frequency unique \emph{k}-mers.
    \item Repeats create high-frequency peaks.
    \item Tools like \texttt{GenomeScope} model errors, repeats, heterozygosity.
\end{itemize}

% ----------------------------------------------------------
\section*{6. Discussion Starters}
\begin{itemize}
    \item Differences among sequencing generations.
    \item FASTA vs FASTQ usage.
    \item How trimming affects \emph{k}-mer spectra.
    \item Why repeats cause high-frequency \emph{k}-mers.
    \item How the Poisson distribution relates to coverage.
\end{itemize}

\newpage

% ==========================================================
% IN-CLASS EXERCISES
% ==========================================================
\section*{In-Class Exercises}

% ------------------------------
\subsection*{Exercise 1 --- Compare Sequencing Technologies}
Using the table in Section 2, decide which platform you would use for:
\begin{enumerate}[(a)]
    \item genome assembly,
    \item variant calling,
    \item metagenomics.
\end{enumerate}
Discuss trade-offs in read length, accuracy, speed, throughput, and biases.

\begin{solution}
\textbf{(a) Assembly:} \textbf{PacBio HiFi} or \textbf{Nanopore}. Long reads resolve repeats and SVs. HiFi pairs long length with very high accuracy; Nanopore provides ultra-long reads when needed (telomere-to-telomere).\\[4pt]
\textbf{(b) Variant calling:} \textbf{Illumina} or \textbf{PacBio HiFi}. Illumina excels for SNPs/indels due to low error rates; HiFi adds long-range context and strong SV detection.\\[4pt]
\textbf{(c) Metagenomics:} \textbf{Illumina} (deep, accurate profiling) or \textbf{Nanopore} (rapid, long reads; helpful for assembly in complex communities). 
\end{solution}

% ------------------------------
\subsection*{Exercise 2 --- Quality Score Interpretation}
Given this FASTQ quality string:
\begin{verbatim}
@@@DDDDFFFFFGHIJ
\end{verbatim}
\begin{itemize}
    \item Convert several characters to Phred quality values.
    \item Identify low-quality read regions.
    \item Discuss how trimming would affect downstream \emph{k}-mer counting.
\end{itemize}

\begin{solution}
\textbf{ASCII+33 mapping (examples):} @ (64) $\to Q=31$; D (68) $\to Q=35$; F (70) $\to Q=37$; G (71) $\to Q=38$; H (72) $\to Q=39$; I (73) $\to Q=40$; J (74) $\to Q=41$.\\
\textbf{Interpretation:} All values are high (Q31--41), so no clear low-quality tail in this toy string.\\
\textbf{Trimming rationale:} In real data, low-quality 3' tails create spurious unique \emph{k}-mers (singletons), distorting histograms and inflating genome-size estimates. Trimming reduces this noise.
\end{solution}

% ------------------------------
\subsection*{Exercise 3 --- k-mer Counting Thought Experiment}
Sequence:
\begin{verbatim}
ATGATGCT
\end{verbatim}
Tasks:
\begin{itemize}
    \item List all 4-mers.
    \item Count their frequencies.
    \item Predict how a sequencing error or repeat affects the histogram.
\end{itemize}

\begin{solution}
\textbf{4-mers (sliding window):} ATGA, TGAT, GATG, ATGC, TGCT. Each occurs once.\\[4pt]
\textbf{Effect of one sequencing error:} Typically creates novel singletons (low-frequency \emph{k}-mers), adding a left-tail to the histogram.\\[4pt]
\textbf{Effect of a repeat:} Increases counts proportionally (e.g., doubling the region gives each 4-mer count 2), creating higher-frequency peaks or secondary peaks.
\end{solution}

% ------------------------------
\subsection*{Exercise 4 --- Mini Case: Genome Size Estimation}
Toy dataset:
\begin{center}
\begin{tabular}{l c}
\texttt{ATG} & 20 \\
\texttt{TGA} & 19 \\
\texttt{GAT} & 22 \\
\texttt{ATC} & 1 \\
\texttt{TCA} & 1 \\
\end{tabular}
\end{center}

Tasks:
\begin{itemize}
    \item Identify likely sequencing errors.
    \item Estimate approximate coverage from the main peak.
    \item Explain how GenomeScope models errors, repeats, heterozygosity.
\end{itemize}

\begin{solution}
\textbf{Likely errors:} ATC and TCA (singletons).\\[4pt]
\textbf{Coverage estimate:} Main cluster $\approx 20\times$.\\[4pt]
\textbf{GenomeScope intuition:} Fits a mixture to the k-mer spectrum.\\
Errors $\Rightarrow$ leftmost tail (low counts).\\
Repeats $\Rightarrow$ peaks at multiples of coverage ($2C, 3C,\dots$).\\
Heterozygosity $\Rightarrow$ sub-peak near $C/2$ (e.g., $\sim 10$ here).
\end{solution}

\end{document}