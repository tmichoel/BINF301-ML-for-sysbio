\documentclass[11pt,a4paper]{article}

% ===============================
% Packages
% ===============================
\usepackage[margin=2.2cm]{geometry}
\usepackage{setspace}
\usepackage{titlesec}
\usepackage{enumerate}
\usepackage{tabularx}
\usepackage{booktabs}
\usepackage{amsmath}
\usepackage{hyperref}
\usepackage{fancyhdr}
\usepackage{parskip}
\usepackage{lmodern}
\usepackage{xcolor}
\usepackage{environ}
\usepackage{verbatim}
\usepackage{tikz}
\usetikzlibrary{arrows.meta,positioning,calc,decorations.pathmorphing}

% ===============================
% Toggle: student vs instructor
% ===============================
\newif\ifsolutions
\solutionsfalse    % <-- Student version (no solutions)
%\solutionstrue       % <-- Instructor version (solutions visible)

% ===============================
% Styles (Style 1 — Clean Minimalist)
% ===============================
\definecolor{instrblue}{RGB}{0,70,160}
\definecolor{edgeblue}{RGB}{30,90,180}

% Diagram styles
\tikzset{
  read/.style={rectangle, rounded corners=2pt, draw=black!60, fill=black!5, inner sep=3pt, minimum width=18pt, minimum height=12pt, font=\footnotesize},
  node/.style={circle, draw=black!60, fill=black!5, inner sep=1.8pt, minimum size=16pt, font=\footnotesize},
  edge/.style={-Latex, very thin, draw=edgeblue},
  altedge/.style={-Latex, very thin, draw=black!60, dashed},
  pathhl/.style={-Latex, semithick, draw=black}, % highlight path (monochrome friendly)
  err/.style={-Latex, thin, draw=black, dashed}, % shows erroneous or low-confidence edges
  tip/.style={-Latex, thin, draw=black!60, densely dotted}
}

% ===============================
% Robust solution environment
% ===============================
\NewEnviron{solution}{%
  \ifsolutions
    \par\medskip
    \noindent\textbf{\color{instrblue}Solution. }\color{instrblue}\BODY
    \par\medskip\color{black}
  \fi
}

\newcommand{\instrnote}[1]{\ifsolutions{\color{instrblue}\emph{[#1]}}\fi}

% ===============================
% Header / Footer
% ===============================
\pagestyle{fancy}
\fancyhf{}
\lhead{Sequencing \& Assembly — Lecture 3}
\rhead{BINF301}
\rfoot{\thepage}
\ifsolutions
  \lhead{Instructor Version — Solutions Included}
\fi

% Section spacing
\titlespacing*{\section}{0pt}{10pt}{4pt}
\titlespacing*{\subsection}{0pt}{7pt}{3pt}

% ===============================
\begin{document}

\begin{center}
    {\LARGE \textbf{Lecture 4: Long-Read Assembly}}\\[6pt]
    {\Large \textbf{Student Handout \& In-Class Exercises}}\\[10pt]
    \textbf{Course:} BINF301 — Computational Biology \\
    \textbf{Instructor:} Tom Michoel \\
    \textbf{Date:} 26/01/2026\\
    \textbf{Created with Copilot}
\end{center}

\vspace{0.5cm}

\section{Overview}

This handout summarizes the key concepts from \textbf{Lecture 4: Long-read Assembly}. Topics covered include:

\begin{itemize}
    \item Why De Bruijn graphs are not ideal for long-read data
    \item Overlap-based long-read assembly strategies
    \item Overview of tools: \textbf{Canu, Flye, HiCanu, HiFiAsm}
    \item Haplotype phasing with long reads
    \item Scaffolding using Hi-C contact maps
    \item Assembly polishing and contamination detection
\end{itemize}

\begin{solution}
Lecture 4 emphasizes why DBG-based assembly depends on saturated $k$-mer coverage, which long-read datasets typically lack due to lower read count. Therefore, modern assemblers rely on overlap graphs.
\end{solution}

% ------------------------------------------------------------------

\section{Why Long Reads Break the De Bruijn Graph Assumption}

De Bruijn graph assembly assumes most genomic $k$-mers appear multiple times in the read set. Long-read datasets (PacBio, Nanopore) have:

\begin{itemize}
    \item \textbf{Lower read count} for same coverage
    \item \textbf{High noise} (for earlier long-read technologies)
\end{itemize}

Thus, $k$-mer coverage becomes non-uniform and DBG-based assembly becomes unreliable.

\begin{solution}
Long reads solve repeats via read length, so the assembly strategy should preserve long-range information, not break reads into small $k$-mers.
\end{solution}

% ------------------------------------------------------------------

\section{Noisy Long-read Data and Assembly Approaches}

Early long reads had error rates of 10--20\%, making overlap detection difficult.

Common strategies:

\begin{itemize}
    \item \textbf{Hybrid correction:} use accurate short reads to correct long reads
    \item \textbf{Hierarchical correction:} repeated overlap--correct cycles (e.g., Canu)
    \item \textbf{Direct overlapping:} apply approximate matching (e.g., minimizers, MinHash)
\end{itemize}

\begin{solution}
Hybrid correction was popular during early Nanopore/PacBio development, but is less needed with modern HiFi reads.
\end{solution}

% ------------------------------------------------------------------

\section{Canu}

Canu is a long-read assembler derived from the Celera assembler.

It operates in three phases:

\begin{enumerate}
    \item \textbf{Correction} -- detect overlaps, estimate corrected length, output corrected reads
    \item \textbf{Trimming} -- identify unsupported regions and remove them
    \item \textbf{Assembly} -- build the final overlap graph and output contigs
\end{enumerate}

Canu uses the MHAP (MinHash Alignment Process) algorithm for fast overlap detection:

\begin{itemize}
    \item Decompose reads into $k$-mers
    \item Hash $k$-mers and select \emph{min-mers}
    \item Fraction of shared min-mers approximates sequence similarity
\end{itemize}

\begin{solution}
MHAP dramatically reduces the cost of all-vs-all comparisons, which is the primary bottleneck in OLC assemblers.
\end{solution}

% ------------------------------------------------------------------

\section{Flye}

Flye uses a \textbf{repeat graph} rather than a classical overlap graph.

Key ideas:

\begin{itemize}
    \item Build \textbf{disjointigs}: arbitrary merges of overlapping fragments
    \item Use disjointigs to form a draft assembly graph
    \item Distinguish:
    \begin{itemize}
        \item \textbf{Bridged repeats} -- some read spans the repeat
        \item \textbf{Unbridged repeats} -- resolved through subtle sequence differences
    \end{itemize}
    \item Output final contigs after graph simplification
\end{itemize}

\begin{solution}
Flye’s repeat graphs are robust for genomes with complex repeat structures, especially microbial and eukaryotic genomes.
\end{solution}

% ------------------------------------------------------------------

\section{HiFi Reads and Assemblers (HiCanu, HiFiAsm)}

\subsection{HiCanu}

Optimized for high-accuracy PacBio HiFi reads.

Key features:

\begin{itemize}
    \item Homopolymer compression before overlap detection
    \item Overlap-based trimming
    \item Error correction using read pileups
\end{itemize}

\begin{solution}
HiCanu produces extremely clean contigs because HiFi reads resolve many error modes during overlap correction.
\end{solution}

\subsection{HiFiAsm}

A haplotype-aware assembler that:

\begin{itemize}
    \item Performs all-vs-all overlaps
    \item Identifies \textbf{informative SNP positions}
    \item Groups reads into haplotypes using consistency checks
    \item Builds a string graph where haplotypes appear as ``bubbles''
\end{itemize}

\begin{solution}
HiFiAsm is the current state-of-the-art for diploid HiFi genome assembly due to its haplotype resolution.
\end{solution}

% ------------------------------------------------------------------

\section{Haplotype Phasing}

Diploid and polyploid organisms have heterozygous positions that create bubbles in assembly graphs.

Phasing strategies include:

\begin{itemize}
    \item Read-based phasing (HiFiAsm)
    \item Trio-binning using parental data
    \item Hi-C based chromosome-scale phasing
\end{itemize}

\begin{solution}
Long reads often contain enough SNPs to directly assign them to haplotypes, making phasing far more accurate than short-read methods.
\end{solution}

% ------------------------------------------------------------------

\section{Scaffolding Using Hi-C}

Hi-C provides chromosome-scale contact information:

\begin{itemize}
    \item Contigs with strong Hi-C link density likely belong to the same chromosome
    \item Tools such as SALSA2 and YaHS construct chromosome-scale scaffolds
    \item Hi-C maps also reveal misassemblies (disruptions in the diagonal contact pattern)
\end{itemize}

\begin{solution}
A good Hi-C contact map shows a strong diagonal; off-diagonal blocks often indicate structural errors.
\end{solution}

% ------------------------------------------------------------------

\section{Polishing and Decontamination}

After assembly:

\begin{itemize}
    \item \textbf{Polishing:} tools such as \texttt{Pilon} improve base accuracy using mappings
    \item \textbf{Decontamination:} BlobTools, NCBI FCS detect foreign sequences
    \item \textbf{Organellar assembly:} tools such as MitoHiFi and OATK target mitochondrial/chloroplast genomes
\end{itemize}

\begin{solution}
Decontamination is vital for metagenome-derived samples and field-collected specimens.
\end{solution}

% ------------------------------------------------------------------

\newpage

\section{In-Class Exercises}

\subsection*{Exercise 1: MinHash Overlaps}

A long read is decomposed into the following 6 $k$-mers:

\[
\{AATCG, ATCGT, TCGTA, CGTAC, GTACG, TACGA\}.
\]

The min-hash function selects the lexicographically smallest $k$-mer as the signature.

\begin{enumerate}
    \item Compute the min-mer for the read.
    \item Determine whether two reads overlap if they share the same min-mer.
\end{enumerate}

\begin{solution}
The smallest $k$-mer is \texttt{AATCG}. Two reads sharing this min-mer are likely to overlap, but additional min-mers or sketches are typically needed for robust detection.
\end{solution}

% ----

\subsection*{Exercise 2: Repeat Graph Reasoning}

You observe two repeat copies in a Flye graph: one bridged, one not.

\begin{enumerate}
    \item Explain how Flye resolves the bridged repeat.
    \item Explain how Flye resolves the unbridged repeat.
\end{enumerate}

\begin{solution}
Bridged repeats are resolved by direct read traversal. Unbridged repeats are resolved using small sequence differences and consistency between disjointigs.
\end{solution}

% ----

\subsection*{Exercise 3: Hi-C Misassembly Detection}

Given a Hi-C contact heatmap with a disrupted diagonal:

\begin{enumerate}
    \item Interpret the meaning of an abrupt diagonal break.
    \item Suggest a repair strategy.
\end{enumerate}

\begin{solution}
A diagonal break indicates a misjoin between contigs. Tools like SALSA2 or YaHS can break and reassemble the problematic junction based on Hi-C support.
\end{solution}

% ----

\subsection*{Exercise 4: Haplotype Bubbles}

You see a bubble in a HiFiAsm graph representing two possible paths.

\begin{enumerate}
    \item Identify what genomic feature this represents.
    \item Describe a criterion to decide which path belongs to haplotype A vs B.
\end{enumerate}

\begin{solution}
This represents heterozygosity. Reads supporting each variant (SNPs, small indels) determine the haplotype assignment.
\end{solution}

% ----

\subsection*{Exercise 5: Homopolymer Compression}

Explain why homopolymer compression improves overlap detection for HiFi reads.

\begin{solution}
Most HiFi read errors occur in homopolymer length estimation. Compressing runs (e.g., AAAA → A) removes this error source before alignment.
\end{solution}

% ------------------------------------------------------------------

\end{document}